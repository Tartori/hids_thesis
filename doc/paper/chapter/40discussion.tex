
\chapter{Discussion}
\label{sec:Discussion}

The designed and implemented \gls{hids} works differently than previous \gls{hids}. Many will argue that because it does not generate \glspl{hash} it is insecure. I argue against that, because finding intrusions always takes risks. If a conventional \gls{hids} finds more intrusions but takes a long time to scan the whole system, it is not useful. Finding an intrusion days or even weeks after it happened might not be interesting. The attacker then has a long time to either hide, move on or extract all the information he is interested in. Finding intrusions in a timely fashion is very important and this is where my implementation shines. Additionally, even if it didn't find the intrusion, the database can be very helpful for forensic investigators to find out what the attacker did. This might help finding similar attacks in the future by adapting the configuration or by chaning other components. The argument this system makes is not even to be the one and only. In its current form this does not make sense. I hevily advise people to use a \gls{nids} or a \gls{hids} with other focusses. I also advise people to use a \gls{hids} on file basis which uses \glspl{hash} for highly critical parts of the system. However, if only a hash based \gls{hids} is used, then many attacks will be found to late, or not at all. Another benefit that this implementation has is that it runs on any system. \gls{tsk} runs on Linux, OsX and Windows, so does python. By using those components, this system should work on any of those systems as well.

During my work I realized the system and I tested it with modified data. However, the system has not yet been used in a productive environment and has not yet detected an intrusion. This means that it can not be said without doubt if it would work. The main reason why it has not been tested is that the time ran out. Besides that there were ethical and judical conserns of creating a host that is easily exploitable just so that it can be attacked. This would lead to criminals gaining access to a host to do their work which is not in my interest. Even if my system would find their intrusions, it is still possible that they can abuse the host for some time. Additionally, it is possible that they would use an attack which my system can not detect. This would mean that they have access to the host for a prolonged time. The system would needs to be tested in a live environment where attacks naturally happen. Sadly, I did not have access to such a system. 

