
\chapter{Discussion}
\label{sec:Discussion}


The designed and implemented \gls{fids} works differently than other \gls{hids} that rely on \gls{fim}. Many will argue that because it does not generate \glspl{hash}, it is insecure. This is not true because finding intrusions always takes risks. If a conventional \gls{hids} finds more intrusions but takes a long time to scan the whole system, it is not useful. Finding intrusions days or even weeks after it happened might not be interesting. The attacker then has a long time to either hide, move on, or extract all the information he is interested in. Finding intrusions in a timely fashion is very important, and this is where my implementation shines. Additionally, even if it didn't find the intrusion, the database can be beneficial for forensic investigators to find out what the attacker did. This might help to find similar attacks in the future by adapting the configuration or by changing other components. The argument this system makes is not even to be the one and only. In its current form, this does not make sense. I heavily advise people to use a \gls{nids} or a \gls{hids} with other focus. I also advise people to use a \gls{hids} with \gls{fim} which uses \glspl{hash} for highly critical parts of the system. However, if only a hash-based \gls{hids} is used, then many attacks are found too late, or not at all. However, this risk-based approach is not perfect. The decision to not rely on hashing results in a considerably weakened reliability. An attacker can change the \gls{fs} attributes back to the original value \cite{changing:attributes}. 

During my thesis I created the system and I tested it by modifying data. However, the system has not yet been used in a productive environment and has not yet detected live intrusions. This means that it can not be said without doubt if it would work. The main reason why it has not been tested is that the time ran out. The system would needs to be tested in a live environment where attacks naturally happen. Sadly, I did not have access to such a system. 

The \gls{fids} uses \gls{tsk} to analyze \glspl{fs}. Thus, it is reliant on the support of \gls{tsk} for the different \glspl{fs}. While \gls{tsk} supports many \glspl{fs}, it does not support all of them. Some \glspl{fs} are not yet supported which means that the \gls{fids} cannot directly evaluate them. It is possible to extend \gls{pytsk} to support more \glspl{fs} but this is not yet easilz doable using \gls{fids}. 

However, the \gls{fids} can be used for intrusion detection and for investigation. The output of the timeline creator can be very valuable for an investigator, because it gives a more detailed view on what happened on the \gls{fs} than what is available by using the current state of the host. With the \gls{fids} it is possible to create a timeline which can be viewed further back without losing accuracy. 
