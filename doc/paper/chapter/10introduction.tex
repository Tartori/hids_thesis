\chapter{Introduction}

Attacks on computer systems have become highly prevalent. This situation resulted in the creation of many tools that support the securing of said computer systems. Those tools have different purposes and are grouped into different families. Some of those families are responsible for keeping any malicious activity outside of the network while others try to find out whenever something manages to come through. Those types of system are named \gls{ids}. An \gls{ids} generally works by tracking and evaluating some form of activity from hosts and networks and then tries to find \glspl{anomaly}. They operate by comparing the current activity to some configuration which usually contains a form of a whitelist and blacklist. Found \glspl{anomaly} are then alerted or logged such that investigators and system administrators can evaluate the \gls{anomaly} and depending on the result of the evaluation, take action. \cite{hidsnids}

There are two flavors of \gls{ids}, one, which deals with data from the network, named \gls{nids}, and another which evaluates data from the host system, named \gls{hids}. \gls{nids} are used to detect unusual behavior of the network like communication from hosts which usually don't communicate or suspiciously large amounts of data that is transferred. \gls{hids}, on the other hand, are used to detect \glspl{anomaly} on a host. This process works by detecting changes like which processes are started or when the \gls{fs} is changed. Both types of \gls{ids} have various advantages, and they should be used in conjunction for best results. \cite{hidsnids}

This thesis is about the writing of a \gls{hids} by analyzing changes on the \gls{fs} through \gls{fim}. As already mentioned, a \gls{hids} operates on the host machine and detects \glspl{anomaly} by comparing resources available on the host. For \gls{fim} a \gls{hids} usually calculates a \gls{hash} for each file and compares them to previous executions. If the output of the \gls{hash} changes, then the file has been altered. If this alteration is unexpected and detected as an \gls{anomaly} by comparing it to a configuration, it is alerted. This approach has one weakness. The calculation of a \glspl{hash} takes time.

Additionally, to calculate the hash function, the \gls{hids} needs to read the whole content of the relevant files. Because those files are stored on some \gls{storagemedia}, it takes even more time for the whole process of reading the file from the \gls{fs} and calculating the hash \cite{hash:slow, hash:speed}. This issue is, from a historical perspective, not relevant, as it was efficient enough that the entire \gls{fs} could be hashed in a small amount of time. However, \glspl{storagemedia} grew, and with it, the amount of data on a server \cite{bruce:imaging}. With that, the calculation of \glspl{hash} needs more time. So much indeed, that traditional \gls{hids} can't scan big systems within a valuable amount of time anymore.

There is another way. Instead of reading the entire file and calculating the hashes for each, the \gls{fs} already contains some information that can be valuable for \gls{fim}. The \gls{fs} contains much \gls{metadata}. Things like modification date, permissions, and file size. Accessing this type of information is cheap, and it can be achieved without touching any contents within the file. This data can be used for \gls{fim} and finally, to find \glspl{anomaly} \cite{inode}. While not as reliable as hashing, the speed gain can justify sacrificing a small fraction of reliability. This risk has to be considered because finding an intrusion fast can be the difference between an attempted intrusion and a successful attack with possible exfiltration of data. 

Another advantage that this \gls{hids} has is the built-in support for forensic investigations. Sometimes it is not sufficient to find intrusions, because every \gls{ids} can miss some intrusions which then need manual investigation. For those investigations to be successful, more data is better because it makes it easier to reconstruct the attack. This way, they can evaluate what has happened and assess the risk. Most \gls{hids} solutions don't offer much help in this regard. The solution developed within this thesis, however, has this idea of a forensic investigation built-in.

