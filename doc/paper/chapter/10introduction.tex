\chapter{Introduction}

For modern businesses it is imperative that their computer systems are secured. This way they can focuss on their busines without worrying about the computer network. To support the securing of their computer system, there are many tools, which can be grouped into different types of families. Those families have multiple different purposes, some are responsible to keep any malicious activity outside of the network while others try to find out whenever something manages to come through. Those type of system are named \gls{ids}. An \gls{ids} generally works by tracking and evaluating some form of activity from hosts and networks and then tries to find \glspl{anomaly}. This is usually done by comparing the current activity to some sort of configuration which usually contains a form of a whitelist and blacklist. Found \glspl{anomaly} are then alerted or logged such that investigators and system administrators can evaluate the anomaly and maybe take action. 

There are two flavours of \gls{ids}, one, which deals with data from the network, named \gls{nids}, and another which evaluates data from the host system, named \gls{hids}. \gls{nids} are used to detect unusual behavior of the network like communication from hosts which usually don't communicate and suspiciously large amounts of data. \gls{hids}, on the other hand, are used to detect \glspl{anomaly} on a host. This process works by detecting changes in the way processes are used or when the \gls{fs} is changed. Both types of \gls{ids} have different advantages, and they should be used in conjunction for best results. \cite{needed}

This thesis is about writing of a \gls{hids} that operates on the \gls{fs}. As already mentioned a \gls{hids} operates on the host machine and detects \glspl{anomaly} by comparing resources available on the host. For detecting changes to the \gls{fs} a \gls{hids} usually generates \glspl{hash} and compares them to previous executions. If the \gls{hash} changes, then the file has been altered. If this alteration is detected as a \gls{anomaly}, it is alerted. This approach has one weakness. The calculation of a \gls{hash} takes time. Specifically, to calculate the hash the \gls{hids} needs to read the whole content of the relevant files. Because those files are stored on some storage medium, it can actually take some additional time to read the file from the \gls{fs} and then calculate the hash \cite{hash:slow, hash:speed}. This is from a historical perspective not relevant as it was efficient enough that the entire \gls{fs} could be hashed in a small amount of time. However, storage media grew, and with it, the amount of data on a server \cite{bruce:imaging}. With that, the calculation of \glspl{hash} needs more time. So much indeed, that traditional \gls{hids} can't scan big systems within a valuable amount of time anymore.

There is another way though. Instead of reading the entire file and calcualating the hashes for each, the \gls{fs} already contains some information that could be used to find \glspl{intrusion}. The \gls{fs} contains a lot of \gls{metadata}. Things like modification date, permissions and file size. Accessing this type of information is cheap and it can be done without touching any type of content within the file. This data can be used to find anomalies, while not as reliable as hashing, the speed gain can justify sacrificing a little bit of reliability. This risk has to be considered, because finding an intrusion fast can be the difference between an attempted intrusion and an exfiltration of all the busines relevant data. \cite{inode}

Another advantage that this \gls{hids} has is the built in support for forensic investigations. Sometimes it is not sufficient to find intrusions, because every \gls{ids} can miss some intrusions which usually are investigated. For those investigations to be successful, more data is better because it makes it easier to reconstruct the attack. This way, they can evaluate what has happened and assess the risk. Most \gls{hids} solutions don't offfer much help in this regard. The solution developed within this thesis, however, has this idea of a forensic investigation built in.

\section{Drawbacks}

This implementation doesn't calculate any \glspl{hash}. This results in some blind spots. The \gls{fs} \gls{metadata} can be changed after an infection and without calculating a \gls{hash} those changes will not be found. \cite{changing:attributes} This risk needs to be weighed against the gained speed and the benefit of finding intrusions fast. This implementation does not claim to be the ultimate tool which finds all intrusions; rather, it is one piece of the puzzle. Further information about attack scenarios and mitigations are available in section \ref{sec:attack_scenarios} and section \ref{sec:mittigations}.
