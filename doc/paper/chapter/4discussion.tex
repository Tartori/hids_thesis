
\chapter{Discussion}
\label{sec:Discussion}

The designed and implemented \gls{hids} works differently than previous \gls{hids}. Many will argue that because it does not generate \glspl{hash} it is insecure. I argue against that, because finding intrusions always takes risks. If a conventional \gls{hids} finds more intrusions but takes a long time to scan the whole system, it is not useful. Finding an intrusion days or even weeks after it happened might not be interesting. The attacker then has a long time to either hide, move on or extract all the information he is interested in. Finding intrusions in a timely fashion is very important and this is where my implementation shines. Additionally, even if it didn't find the intrusion, the database can be very helpful for forensic investigators to find out what the attacker did. This might help finding similar attacks in the future by adapting the configuration or by chaning other components. The argument this system makes is not even to be the one and only. In its current form this does not make sense. I hevily advise people to use a \gls{nids} or a \gls{hids} with other focusses. I also advise people to use a \gls{hids} on file basis which uses \glspl{hash} for highly critical parts of the system. However, if only a hash based \gls{hids} is used, then many attacks will be found to late, or not at all. Another benefit that this implementation has is that it runs on any system. \gls{tsk} runs on Linux, OsX and Windows, so does python. By using those components, this system should work on any of those systems as well.

During my work I realized the system and I tested it with modified data. However, the system has not yet been used in a productive environment and has not yet detected an intrusion. This means that it can not be said without doubt if it would work. The main reason why it has not been tested is that the time ran out. Besides that there were ethical and judical conserns of creating a host that is easily exploitable just so that it can be attacked. This would lead to criminals gaining access to a host to do their work which is not in my interest. Even if my system would find their intrusions, it is still possible that they can abuse the host for some time. Additionally, it is possible that they would use an attack which my system can not detect. This would mean that they have access to the host for a prolonged time. The system would needs to be tested in a live environment where attacks naturally happen. Sadly, I did not have access to such a system. 

\section{Future Work}
\label{sec:future:work}

It would make sense to extend this \gls{hids} to extend past the \gls{fs}. It would be great if the scanner can also scan the processes and network connections. Not only would this data be important to finding intrusions by the system, but it could also give investigators much more information. Information which they are currently mostly missing.

The system should also be extensively tested. For this it needs to run for a prolonged time in a productive system. The output and the found intrusions would then need to be compared to other system to gain important information about how fast and how many intrusions are found using this system. 

One other path that could be improved on would be the investigator. Currently it operates only on a configuration which someone must write. It would be great to autonomously analyse the scanner output and find anomalies on them. This could be done if a lot of data has already been collected on many different types of hosts. The types could then be grouped and an algorithm could detect similar patterns to find simmilar intrusions. 

Generally, the field of finding intrusions has a lot of opportunities for research. This system can be one part of an extensive system that checks for intrusions. Especially since disk sizes and data usage is still growing it is important to have such a system that can find intrusions fast.

It would also be interesting to add the output of the scanner to a super timeline. It would help seeing what happened on the host at any point in time. 
