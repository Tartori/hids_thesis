\documentclass[
	a4paper,					% paper format
	10pt,							% fontsize
	twoside,					% double-sided
	openright,				% begin new chapter on right side
	notitlepage,			% use no standard title page
	parskip=half,			% set paragraph skip to half of a line
]{scrreprt}					% KOMA-script report

\raggedbottom
\KOMAoptions{cleardoublepage=plain}			% Add header and footer on blank pages

\usepackage{csquotes}
\usepackage[hidelinks]{hyperref}
\usepackage{color}

% Code Segments
\usepackage{listings}

\usepackage[english]{babel}										% english hyphenation
\usepackage[utf8]{inputenc}  							% Unix/Linux - load extended character set (ISO 8859-1)
%\usepackage[ansinew]{inputenc}  							% Windows - load extended character set (ISO 8859-1)
\usepackage{fancyhdr}													% simple manipulation of header and footer
\usepackage{etoolbox}													% color manipulation of header and footer
\usepackage{graphicx}                      		% integration of images
\usepackage{float}														% floating objects
\usepackage{caption}													% for captions of figures and tables
\usepackage{booktabs}													% package for nicer tables
\usepackage{tocvsec2}													% provides means of controlling the sectional numbering
\usepackage{tabularx}
%---------------------------------------------------------------------------

% Set up page dimension
%---------------------------------------------------------------------------
\usepackage{geometry}
\geometry{
	a4paper,
	left=28mm,
	right=15mm,
	top=30mm,
	headheight=20mm,
	headsep=10mm,
	textheight=242mm,
	footskip=15mm
}


% Compact Itemize:
%---------------------------------------------------------------------------
\newenvironment{compactitemize}
{ \begin{itemize}
    \setlength{\itemsep}{0pt}
    \setlength{\parskip}{0pt}
    \setlength{\parsep}{0pt}     }
{ \end{itemize}                  }
\newenvironment{compactenumerate}
{ \begin{enumerate}
    \setlength{\itemsep}{0pt}
    \setlength{\parskip}{0pt}
    \setlength{\parsep}{0pt}     }
{ \end{enumerate}  				 }

\RequirePackage{color}                          % Color (not xcolor!)
\definecolor{linkblue}{rgb}{0,0,0.8}            % Standard
\definecolor{darkblue}{rgb}{0,0.08,0.45}        % Dark blue
\definecolor{bfhgrey}{rgb}{0.41,0.49,0.57}      % BFH grey
\definecolor{linkcolor}{rgb}{0,0,0.8}  
\definecolor{darkblue}{rgb}{0,.2,.4}
\definecolor{darkgray}{rgb}{.4,.4,.4}
\definecolor{purple}{rgb}{0.65, 0.12, 0.82}
\definecolor{brown}{rgb}{.4,.4,.3}
\definecolor{darkred}{rgb}{.6,0,0}
\definecolor{linenumbergray}{rgb}{.6,.6,.6}   			% Blue for the web- and cd-version!
%\definecolor{linkcolor}{rgb}{0,0,0}        			% Black for the print-version!


\usepackage{colortbl}
\usepackage{longtable}
\usepackage{lscape}


\newcolumntype{L}[1]{>{\raggedright\let\newline\\\arraybackslash\hspace{0pt}}m{#1}}
\newcolumntype{C}[1]{>{\centering\let\newline\\\arraybackslash\hspace{0pt}}m{#1}}
\newcolumntype{R}[1]{>{\raggedleft\let\newline\\\arraybackslash\hspace{0pt}}m{#1}}



% Sequence Diagram
\usepackage{geometry}
\usepackage{pgf-umlsd}
\usetikzlibrary{calc}

% Glossary
\usepackage[toc,section=section]{glossaries}
\makeglossaries
\usepackage{glossaries}

\newacronym{cli}{CLI}{comand line interface}
\newacronym{sha256}{SHA-256}{Secure Hashing Algorithm - 256}

\newacronym{ids}{IDS}{intrusion detection system}
\newacronym{apt}{APT}{advanced persistant threat}
\newacronym{fim}{FIM}{file integrity monitoring} 
\newacronym{hids}{HIDS}{host-based intrusion detection system} 
\newacronym{nids}{NIDS}{network-based intrusion detection system} 
\newacronym{nist}{NIST}{National Institute of Standards and Technology}
\newacronym{owasp}{OWASP}{Open Web Application Security Project}
\newacronym{it}{IT}{Information Security}
\newacronym{tb}{TB}{Terrabyte}
\newacronym{mb}{MB}{Megabyte}
\newacronym{kb}{KB}{Kilobyte}
\newacronym{itsec}{ITSec}{\gls{it} Security}
\newglossaryentry{yaml}{name=YAML, description={YAML Ain't Markup Language}}
\newglossaryentry{regex}{name=regex, description={Regular Expressions are a standard way to find certain patterns in a string.}}
\newacronym{tsk}{TSK}{The Sleuth Kit}
\newacronym{tls}{TLS}{Transport Layer Security}
\newacronym{api}{API}{Application Programming Interface}
\newacronym{json}{JSON}{JavaScript Object Notation}
\newacronym{loa}{LoA}{Level of Assurance}
\newacronym{hdd}{HDD}{Hard Disk Drive}
\newacronym{ssd}{SSD}{Solid State Drive}
\newacronym{dbms}{DBMS}{DataBase Management System}
\newacronym{id}{ID}{IDentifier}
\newacronym{uuid}{UUID}{Universaly Unique IDentifier}
\newacronym{cd}{CD}{Compact Disc}
\newacronym{dvd}{DVD}{Digital Versatile Disc}
\newacronym{usb}{USB}{Universal Serial Bus}
\newacronym{ioc}{IoC}{Indicator of Compromise}
\newglossaryentry{sql}{name=SQL, description={A domain specific language used for querying a relational database.}}
\newglossaryentry{unixts}{name=UNIX timestamp, description={The number of seconds since 00:00:00 on the 1st of January 1970}}
\newglossaryentry{hex}{name=hexadecimal, description={A number system with 16 digits. It uses the numbers 0-9 and the Letters A-F.}}


\newglossaryentry{pgp}{type=\acronymtype, name={PGP}, description={Pretty Good Privacy}, first={Pretty Good Privacy (PGP)\glsadd{pgpg}}, see=[Glossary:]{pgpg}}
\newglossaryentry{}{name=, description={}}
\newglossaryentry{pgpg}{name=PGP, description={Standard for encryption and signatures defined in RFC4880}}
\newglossaryentry{postgres}{name=postgres, description={An opensource database server. It is rather lightweight and heavily used.}}
\newglossaryentry{sqlite}{name=sqlite, description={A very lightweight relational database implemnentation. It does not have a dedicated server but instead writes the database to a file which is written to the local host.}}
\newglossaryentry{opensource}{name=opensource, description={Software where both the source and the software is freely accessible and changable.}}
\newglossaryentry{github}{name=github, description={A platform for \gls{opensource} projects. It is free to use and hosts the source code for many projects.}}
\newglossaryentry{storagemedia}{name=storage media, description={media that is used to store data in a computer system.}}
\newglossaryentry{malware}{name=malware, description={Malware is any program that is designed to harm a computer system. The term includes well known terms like Virus, Worm, Adware, Keylogger, Trojan, etc. Malware usually tries to hide it's traces to achieve longer infection periods of time. }}
\newglossaryentry{collision}{name=collision, plural=collisions, description={Multiple different inputs with the same hash value.}}

\newglossaryentry{anomaly}{
name=anomaly, 
plural=anomalies,
description={An unexpected change in the file system. For a change to be unexpected it needs to be covered by the configuration changes are unexpected only if the configuration says so. Some examples that might cause anomalies, changed rights, new files, deleted files. }}
\newglossaryentry{intrusion}{
name=intrusion, 
plural=intrusions,
description={An unauthorized access to a system or to data. }}
\newglossaryentry{investigation}{
name=investigation, 
plural=investigations,
description={The process of finding out what exactly happened after an incident.}}
\newglossaryentry{hash}{
name=cryptographic hash function, 
plural=cryptographic hash functions,
description={A deterministic one way function that fullfills collision resistance and other cryptographic properties. Implementations are the SHA-2 and SHA-3 families.}}
\newglossaryentry{nonhash}{
name=non-cryptographic hash function, 
plural=non-cryptographic hash functions,
description={A deterministic one way function that does not the hard to achieve properties that make a \gls{hash}. Often used when speed is more important than collision resistance.}}
\newglossaryentry{fs}{
name=file system, 
description={A file system is used to create a layer of abstraction between the hardware of the storage medium and the operating system. There are multiple file system types which are in use with different capabilities. Additionally to the files the file systems keep track of meta data to each file. This meta data includes some timestamps (created, last accessed, etc) and more.}}
\newglossaryentry{pytsk}{name=pytsk3, description={Python bindings for \gls{tsk}}}
\newglossaryentry{metadata}{name=metadata, description={Data that gives information on other data. In this thesis it is mostly used to describe attributes of the \gls{fs} that describes files. Exmaples are creation date and permissions.}}
\newglossaryentry{git}{name=git, description={A distributed version-control system used for source code tracking in software engineering. Nicknamed \'the stupid content tracker\', stands for either \'global information tracker\' or \'goddamn idiotic truckload of s...\'}}








\usepackage{polyglossia}
\setdefaultlanguage{english}
\usepackage[backend=biber, style=ieee]{biblatex}
\addbibresource{thesis.bib}
\usepackage{graphicx}

\usepackage{pgfgantt}

\definecolor{ganttplanned}{RGB}{0,80,200}
\definecolor{ganttplannedopt}{RGB}{50,50,50}
\definecolor{ganttactual}{RGB}{234,187,0}
\definecolor{ganttunplanned}{RGB}{153,0,0}

\newganttchartelement*{plannedmilestone}{%
  plannedmilestone/.style={
	shape=ganttmilestone,
	inner sep=0pt,
	draw=ganttplanned!50!black,
	top color=white,	
	bottom color=ganttplanned!50% 
  },
  plannedmilestone label text=\strut#1,
  plannedmilestone label font=\footnotesize,
  plannedmilestone label node/.style={%
	anchor=east, font=\ganttvalueof{plannedmilestone label font}%
  },%
  plannedmilestone inline label anchor=center,%
  plannedmilestone inline label node/.style={%
	anchor=south, font=\ganttvalueof{plannedmilestone label font}%
  },%
  plannedmilestone left shift = .6,
  plannedmilestone right shift = .4,
  plannedmilestone top shift = .05,
  plannedmilestone height = .6
}
\newganttchartelement*{actualmilestone}{%
  actualmilestone/.style={
	shape=ganttmilestone,
	inner sep=0pt,
	draw=ganttactual!50!black,
	top color=white,	
	bottom color=ganttactual!50% 
  },
  actualmilestone label text=\strut#1,
  actualmilestone label font=\footnotesize,
  actualmilestone label node/.style={%
	anchor=east, font=\ganttvalueof{actualmilestone label font}%
  },%
  actualmilestone inline label anchor=center,%
  actualmilestone inline label node/.style={%
	anchor=south, font=\ganttvalueof{actualmilestone label font}%
  },%
  actualmilestone left shift = .6,
  actualmilestone right shift = .4,
  actualmilestone top shift = .35,
  actualmilestone height = .6
}





\begin{document}
\title{Alternative scalable HIDS with investigation capability}
\date{\today} 
\author{ Julian Stampfli (\texttt{stamj3@bfh.ch}) }
\maketitle
\setcounter{tocdepth}{2}
\tableofcontents
\clearpage

\chapter{Project Management}

\section{Goal}
\label{apdx-sec:goal}
Before the start of the project the following main goal was defined:

Building of an \gls{hids} that detects unauthorized or unusual behaviour on the file system. Compared to traditional \gls{hids} file system integrity checking, it should scale with a lot of data and have the possibility to be used for investigation (retain historic data) built in from the start.

\subsection{Sub goals}

From this primary goal, the following sub goals were defined. 

\subsubsection{Scanning}
The system is capable of scanning the file system for certain properties. The search is done by leveraging the sleuthkit tools. Thus the system is capable of interpreting the results from sleuthkit. It will further analyze them and decide on what to do with the results. Especially importance is given to the finding of differences.

\subsubsection{Recording}
The system records all findings. Including new, changed and deleted files in comparison to an earlier point in time. This recording enables the use of investigation as the evolution of the data can be viewed at any time. This data can also be used for machine learning algorithms to detect anomalies that are out of the scope of this thesis. 

\subsubsection{Evaluation}
The system is capable of evaluating the results by applying predefined rules. Those rules can be adjusted by configuring the system.

It is thinkable that the system analyzes the recordings and makes decisions based on the historical behavior of the specific host and behavior from different similar hosts. This approach is not part of this thesis as it requires much historical data that is not present at the time of this thesis. 

\subsubsection{Alerting}
The system is capable of being run continuously. This capability enables it to find anomalies automatically. The system can report those anomalies by creating alerts. It allows configuration of these alerts.

\subsubsection{Scaling}
The run of the system on a sufficiently big file system completes in an appropriate amount of time. This speed allows the finding of anomalies that appeared recently. Additionally, it allows the storing of more states of the system which results in a her probability of capturing short-lived anomalies for future investigations. 

\subsubsection{Run Time}

The system should be able to complete a full execution on a live system in 15 minutes. This time increases with the disksize and the amount of files. 

\section{Workpackages}

From those goals the workpackages in table \ref{tab:workpackages} were defined. For a better overview they are assigned to categories. The categories are Architecture, Implementation, Validation and Administrative. Architecture is about defining how the system will look like and how it should work. Implementation is the effective implementation work for getting the system to run, this includes configuration and coding. Validation is about testing of the system. Administrative is everything that deals with project management and other workpackages that don't directly influence the system but need to be done.

The ID is a combination of the first letter of the category and a unique index. Administrative is shortened to D due to the conflict with Architecture.

The priority is a value of high, medium and low. 

The workpackages are chrononically ordered. Meaning they should be worked on in approximately the order that they are given. 

\begin{table}[h!]
  \begin{center}
    \caption{Workpackages}
    \label{tab:workpackages}
    \begin{tabular}{c|l|c|l}
      \textbf{ID} & \textbf{Short description} & \textbf{Prio} & \textbf{Category} \\
      \hline
      D00 & Setting up \LaTeX -document & h & Administrative \\
      D01 & Define workpackages and set deadlines & h & Administrative \\
			A00 & Research other \gls{hids} and the sleuthkit tools & h & Architecture \\
			A01 & Decide on a Programming Language & h & Architecture \\
			I00 & Setup the developer environment & h & Implementation \\
			I01 & Add the ability to scan the whole system using sleuthkit & h & Implementation \\
			A02 & Decide which database connectors should be used & h & Architecture \\
			I02 & Add one database connector & h & Implementation \\
			I03 & Implement a recording functionality & h & Implementation \\
			A03 & Decide how the rules should be defined & h & Architecture \\
			I04 & Add template rules and ability to parse them & h & Implementation \\
			I05 & Add functionality to parse output according to rules & h & Implementation \\
			V00 & Verify that the system runs on a big file system & h & Validation \\
			I06 & Add functionality of repeated scans & m & Implementation \\
			A04 & Define which alerting methods make sense & m & Architecture \\ 
			I07 & Add alerting functionality using one method & m & Implementation \\
			V01 & Verify the functionality of the software by changing the system & h & Validation \\
			V02 & Verify the functionality of the software by running it on an infected system & m & Validation \\
			V03 & Verify the alerting of the software by running it on an infectable system & m & Validation \\
			I08 & Add multiple database connectors to different systems & m & Implementation \\
			I09 & Add multiple alerting methods & m & Implementation \\
			A05 & Define how to protect system and configuration from tampering & l & Architecture \\
			I10 & Implement software hardening & l & Implementation \\
			D02 & Finish user documentation & m & Administrative \\
			D03 & Finish project documentation & h & Administrative \\ 
			D04 & Create project presentation & h & Administrative \\
			D05 & Create project poster & m & Administrative \\
			D06 & Create project video & l & Administrative \\
    \end{tabular}
  \end{center}
\end{table}

\section{Planning}

For the planning of this project the following milestones were created. Each coveres multiple workpackages. The mapping can be seen in table \ref{tab:milestones}. The milestones can also be seen in figure \ref{apdx-fig:milestones}. There they are displayed with an assumed and actual finish date.



\begin{table}[h!]
  \begin{center}
    \caption{Milestones}
    \label{tab:milestones}
    \begin{tabular}{c|l|c|l}
      \textbf{ID} & \textbf{Short description} & \textbf{Workpackages} \\
      \hline
			00 & Setup & D00, D01, A00, A01, I00 \\
			01 & Initial functionality & A02, I01, I02, I03 \\
			02 & Rules & A03, I04, I05, V00 \\
			03 & Alerting & A04, I06, I07 \\
			04 & Exhaustive testing & V01, V02, V03 \\
			05 & Usability & I08, I09, D02 \\
			06 & Software Hardening & A05, I10 \\
			07 & Presentation & D02, D03, D04, D05, D06 \\
    \end{tabular}
  \end{center}
\end{table}


\begin{figure}[H]
	\begin{ganttchart}[
		hgrid,
		vgrid,
		x unit=7mm,
		y unit chart=10mm,
		milestone label font = \footnotesize
	]{1}{17}
	\gantttitle{2019}{17}\\
	\gantttitlelist{1,...,17}{1}\\
	
	\ganttplannedmilestone{Setup}{4}
	\ganttactualmilestone{}{0}\\
	\ganttplannedmilestone{Initial functionality}{5}
	\ganttactualmilestone{}{0}\\
	\ganttplannedmilestone{Rules}{7}
	\ganttactualmilestone{}{0}\\
	\ganttplannedmilestone{Alerting}{8}
	\ganttactualmilestone{}{0}\\
	\ganttplannedmilestone{Exhaustive testing}{10}
	\ganttactualmilestone{}{0}\\
	\ganttplannedmilestone{Usability}{13}
	\ganttactualmilestone{}{0}\\
	\ganttplannedmilestone{Software Hardening}{15}
	\ganttactualmilestone{}{0}\\
	\ganttplannedmilestone{Presentation}{17}
	\ganttactualmilestone{}{0}
	\end{ganttchart}
	\caption{Milestones}
	\label{apdx-fig:milestones}
	\end{figure}

\section{Meetings}

\subsection{Template}

Hi Bruce,

Next Meeting:

Discussions:

Next Steps:

Julian


\newpage
\clearpage

\printglossaries

\clearpage

%% Print the bibibliography and add the section to the table of content

\end{document}
