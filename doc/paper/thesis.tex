\documentclass[
	a4paper,					% paper format
	10pt,							% fontsize
	twoside,					% double-sided
	openright,				% begin new chapter on right side
	notitlepage,			% use no standard title page
	parskip=half,			% set paragraph skip to half of a line
]{scrreprt}					% KOMA-script report

\raggedbottom
\KOMAoptions{cleardoublepage=plain}			% Add header and footer on blank pages
\usepackage[utf8]{inputenc}  							% Unix/Linux - load extended character set (ISO 8859-1)

\usepackage{csquotes}
\usepackage[hidelinks]{hyperref}
\usepackage{color}

% Code Segments
\usepackage{listings}
\input{yaml-highlighting.tex}

\usepackage[T1]{fontenc}											% hyphenation of words with �,� and �
\usepackage{textcomp}													% additional symbols
\usepackage{ae}																% better resolution of Type1-Fonts 
\usepackage{fancyhdr}													% simple manipulation of header and footer 
\usepackage{etoolbox}													% color manipulation of header and footer

\usepackage[english]{babel}										% english hyphenation
%\usepackage[ansinew]{inputenc}  							% Windows - load extended character set (ISO 8859-1)
\usepackage{graphicx}                      		% integration of images
\usepackage{float}														% floating objects
\usepackage{caption}													% for captions of figures and tables
\usepackage{booktabs}													% package for nicer tables
\usepackage{tocvsec2}													% provides means of controlling the sectional numbering
\usepackage{tabularx}
%---------------------------------------------------------------------------

% Set up page dimension
%---------------------------------------------------------------------------
\usepackage{geometry}
\geometry{
	a4paper,
	left=28mm,
	right=15mm,
	top=30mm,
	headheight=20mm,
	headsep=10mm,
	textheight=242mm,
	footskip=15mm
}


% Compact Itemize:
%---------------------------------------------------------------------------
\newenvironment{compactitemize}
{ \begin{itemize}
    \setlength{\itemsep}{0pt}
    \setlength{\parskip}{0pt}
    \setlength{\parsep}{0pt}     }
{ \end{itemize}                  }
\newenvironment{compactenumerate}
{ \begin{enumerate}
    \setlength{\itemsep}{0pt}
    \setlength{\parskip}{0pt}
    \setlength{\parsep}{0pt}     }
{ \end{enumerate}  				 }

\RequirePackage{color}                          % Color (not xcolor!)
\definecolor{linkblue}{rgb}{0,0,0.8}            % Standard
\definecolor{darkblue}{rgb}{0,0.08,0.45}        % Dark blue
\definecolor{bfhgrey}{rgb}{0.41,0.49,0.57}      % BFH grey
\definecolor{linkcolor}{rgb}{0,0,0.8}  
\definecolor{darkblue}{rgb}{0,.2,.4}
\definecolor{darkgray}{rgb}{.4,.4,.4}
\definecolor{purple}{rgb}{0.65, 0.12, 0.82}
\definecolor{brown}{rgb}{.4,.4,.3}
\definecolor{darkred}{rgb}{.6,0,0}
\definecolor{linenumbergray}{rgb}{.6,.6,.6}   			% Blue for the web- and cd-version!
%\definecolor{linkcolor}{rgb}{0,0,0}        			% Black for the print-version!


\usepackage{colortbl}
\usepackage{longtable}
\usepackage{lscape}


\newcolumntype{L}[1]{>{\raggedright\let\newline\\\arraybackslash\hspace{0pt}}m{#1}}
\newcolumntype{C}[1]{>{\centering\let\newline\\\arraybackslash\hspace{0pt}}m{#1}}
\newcolumntype{R}[1]{>{\raggedleft\let\newline\\\arraybackslash\hspace{0pt}}m{#1}}


% Package to facilitate placement of boxes at absolute positions
%---------------------------------------------------------------------------
\usepackage[absolute]{textpos}
\setlength{\TPHorizModule}{1mm}
\setlength{\TPVertModule}{1mm}

% Sequence Diagram
\usepackage{geometry}
\usepackage{pgf-umlsd}
\usetikzlibrary{calc}

% Glossary
\usepackage[toc,section=section, acronym]{glossaries}
\makeglossaries
\usepackage{glossaries}

\newacronym{cli}{CLI}{comand line interface}
\newacronym{sha256}{SHA-256}{Secure Hashing Algorithm - 256}
\newacronym{os}{OS}{Operating System}

\newacronym{ids}{IDS}{intrusion detection system}
\newacronym{apt}{APT}{advanced persistant threat}
\newacronym{fim}{FIM}{file integrity monitoring} 
\newacronym{fids}{FIDS}{forensics-based intrusion detection system} 
\newacronym{hids}{HIDS}{host-based intrusion detection system} 
\newacronym{nids}{NIDS}{network-based intrusion detection system} 
\newacronym{nist}{NIST}{National Institute of Standards and Technology}
\newacronym{owasp}{OWASP}{Open Web Application Security Project}
\newacronym{it}{IT}{Information Security}
\newacronym{tb}{TB}{Terrabyte}
\newacronym{gb}{GB}{Gigabyte}
\newacronym{mb}{MB}{Megabyte}
\newacronym{kb}{KB}{Kilobyte}
\newacronym{itsec}{ITSec}{\gls{it} Security}
\newglossaryentry{regex}{name=regex, description={Regular Expressions are a standard way to find certain patterns in a string.}}
\newacronym{tsk}{TSK}{The Sleuth Kit}
\newacronym{tct}{TCT}{The Coroner's Toolkit}
\newacronym{tls}{TLS}{Transport Layer Security}
\newacronym{api}{API}{Application Programming Interface}
\newacronym{json}{JSON}{JavaScript Object Notation}
\newacronym{xml}{XML}{eXtensive Markup Language}
\newacronym{loa}{LoA}{Level of Assurance}
\newacronym{hdd}{HDD}{Hard Disk Drives}
\newacronym{ssd}{SSD}{Solid State Drives}
\newacronym{dbms}{DBMS}{Database Management Systems}
\newacronym{id}{ID}{IDentifier}
\newacronym{uuid}{UUID}{Universaly Unique IDentifier}
\newacronym{cd}{CD}{Compact Discs}
\newacronym{dvd}{DVD}{Digital Versatile Discs}
\newacronym{usb}{USB}{Universal Serial Bus}
\newacronym{ioc}{IoC}{Indicators of Compromise}
\newglossaryentry{sql}{name=SQL, description={A domain specific language used for querying a relational database.}}
\newglossaryentry{unixts}{name=UNIX timestamp, description={The number of seconds since 00:00:00 on the 1st of January 1970}}
\newglossaryentry{hex}{name=hexadecimal, description={A number system with 16 digits. It uses the numbers 0-9 and the Letters A-F.}}

\newglossaryentry{opensource}{name=open source, description={Software where both the source and the software is freely accessible and changable.}}
\newglossaryentry{aide}{name=aide, description={\Gls{opensource} \gls{hids} using \gls{fim} to detect intrusions.}}
\newglossaryentry{tripwire}{name=tripwire, description={Comercial \gls{hids} with good integration in other tools from the tripwire company.}}
\newglossaryentry{samhain}{name=samhain, description={\Gls{opensource} \gls{hids} which mainly uses kernel hook to find intrusions and contains extensive centralization support for configuration and alerting.}}

\newglossaryentry{pgp}{type=\acronymtype, name={PGP}, description={Pretty Good Privacy}, first={Pretty Good Privacy (PGP)\glsadd{pgpg}}, see=[Glossary:]{pgpg}}
\newglossaryentry{yaml}{type=\acronymtype, name={YAML}, description={YAML Ain't Markup Language}, first={YAML Ain't Markup Language (YAML)\glsadd{yyaml}}, see=[Glossary:]{yyaml}}
\newglossaryentry{}{name=, description={}}
\newglossaryentry{pgpg}{name=PGP, description={Standard for encryption and signatures defined in RFC4880}}
\newglossaryentry{yyaml}{name=YAML, description={YAML is a human friendly data serialization
standard for all programming languages.}}
\newglossaryentry{postgres}{name=postgres, description={An ope\gls{opensource} database server. It is rather lightweight and heavily used.}}
\newglossaryentry{sqlite}{name=SQLite, description={A very lightweight relational database implemnentation. It does not have a dedicated server but instead writes the database to a file which is written to the local host.}}
\newglossaryentry{github}{name=github, description={A platform for \gls{opensource} projects. It is free to use and hosts the source code for many projects.}}
\newglossaryentry{storagemedia}{name=storage medium,plural=storage media,description={media that is used to store data in a computer system.}}
\newglossaryentry{malware}{name=malware, description={Malware is any program that is designed to harm a computer system. The term includes well known terms like Virus, Worm, Adware, Keylogger, Trojan, etc. Malware usually tries to hide it's traces to achieve longer infection periods of time. }}
\newglossaryentry{collision}{name=collision, plural=collisions, description={Multiple different inputs with the same hash value.}}

\newglossaryentry{anomaly}{
name=anomaly, 
plural=anomalies,
description={An unexpected change in the file system. For a change to be unexpected it needs to be covered by the configuration changes are unexpected only if the configuration says so. Some examples that might cause anomalies, changed rights, new files, deleted files. }}
\newglossaryentry{intrusion}{
name=intrusion, 
plural=intrusions,
description={An unauthorized access to a system or to data. }}
\newglossaryentry{investigation}{
name=investigation, 
plural=investigations,
description={The process of finding out what exactly happened after an incident.}}
\newglossaryentry{hash}{
name=cryptographic hash function, 
plural=cryptographic hash functions,
description={A deterministic one way function that fullfills collision resistance and other cryptographic properties. Implementations are the SHA-2 and SHA-3 families.}}
\newglossaryentry{nonhash}{
name=non-cryptographic hash function, 
plural=non-cryptographic hash functions,
description={A deterministic one way function that does not the hard to achieve properties that make a \gls{hash}. Often used when speed is more important than collision resistance.}}
\newglossaryentry{fs}{
name=file system, 
description={A file system is used to create a layer of abstraction between the hardware of the storage medium and the operating system. There are multiple file system types which are in use with different capabilities. Additionally to the files the file systems keep track of meta data to each file. This meta data includes some timestamps (created, last accessed, etc) and more.}}
\newglossaryentry{pytsk}{name=pytsk3, description={Python bindings for \gls{tsk}}}
\newglossaryentry{metadata}{name=metadata, description={Data that gives information on other data. In this thesis it is mostly used to describe attributes of the \gls{fs} that describes files. Exmaples are creation date and permissions.}}
\newglossaryentry{git}{name=git, description={A distributed version-control system used for source code tracking in software engineering. Nicknamed \'the stupid content tracker\', stands for either \'global information tracker\' or \'goddamn idiotic truckload of s...\'}}








\newacronym{cwp}{CWP}{Completed Workpackage}
\newacronym{rm}{RM}{Reached Milestones}
\newacronym{hw}{HW}{Hours Worked}
\newacronym{thw}{THW}{Total Hours Worked}
\newacronym{bfh}{BFH}{Berner Fachhochschule - Bern University of Applied Sciences}
\newacronym{bn}{BN}{Bruce Nikkel}
\newacronym{ab}{AB}{Armin Blum}
\newacronym{js}{JS}{Julian Stampfli}




\usepackage{polyglossia}
\setdefaultlanguage{english}
\usepackage[backend=biber, style=ieee]{biblatex}
\addbibresource{thesis.bib}
\usepackage{pgfgantt}


\definecolor{ganttplanned}{RGB}{0,80,200}
\definecolor{ganttplannedopt}{RGB}{50,50,50}
\definecolor{ganttactual}{RGB}{234,187,0}
\definecolor{ganttunplanned}{RGB}{153,0,0}

\newganttchartelement*{plannedmilestone}{%
  plannedmilestone/.style={
	shape=ganttmilestone,
	inner sep=0pt,
	draw=ganttplanned!50!black,
	top color=white,	
	bottom color=ganttplanned!50% 
  },
  plannedmilestone label text=\strut#1,
  plannedmilestone label font=\footnotesize,
  plannedmilestone label node/.style={%
	anchor=east, font=\ganttvalueof{plannedmilestone label font}%
  },%
  plannedmilestone inline label anchor=center,%
  plannedmilestone inline label node/.style={%
	anchor=south, font=\ganttvalueof{plannedmilestone label font}%
  },%
  plannedmilestone left shift = .6,
  plannedmilestone right shift = .4,
  plannedmilestone top shift = .05,
  plannedmilestone height = .6
}
\newganttchartelement*{actualmilestone}{%
  actualmilestone/.style={
	shape=ganttmilestone,
	inner sep=0pt,
	draw=ganttactual!50!black,
	top color=white,	
	bottom color=ganttactual!50% 
  },
  actualmilestone label text=\strut#1,
  actualmilestone label font=\footnotesize,
  actualmilestone label node/.style={%
	anchor=east, font=\ganttvalueof{actualmilestone label font}%
  },%
  actualmilestone inline label anchor=center,%
  actualmilestone inline label node/.style={%
	anchor=south, font=\ganttvalueof{actualmilestone label font}%
  },%
  actualmilestone left shift = .6,
  actualmilestone right shift = .4,
  actualmilestone top shift = .35,
  actualmilestone height = .6
}





\begin{document}


\settocdepth{section}														% Set depth of toc
\pagenumbering{roman}														
%---------------------------------------------------------------------------

\providecommand{\heading}{Alternative scalable HIDS with investigation capability}		%  Insert Title of Thesis here					% Titel der Arbeit aus Datei titel.tex lesen
\providecommand{\versionnumber}{1.0}			%  Hier die aktuelle Versionsnummer eingeben
\providecommand{\versiondate}{June 13. 2019}		%  Hier das Datum der aktuellen Version eingeben				% Versionsnummer und -datum aus Datei version.tex lesen

% Set up header and footer
%---------------------------------------------------------------------------
\makeatletter
\patchcmd{\@fancyhead}{\rlap}{\color{bfhgrey}\rlap}{}{}		% new color of header
\patchcmd{\@fancyfoot}{\rlap}{\color{bfhgrey}\rlap}{}{}		% new color of footer
\makeatother

\fancyhf{}																		% clean all fields
\fancypagestyle{plain}{												% new definition of plain style	
	\fancyfoot[OR,EL]{\footnotesize \thepage} 	% footer right part --> page number
	\fancyfoot[OL,ER]{\footnotesize \heading, Version \versionnumber, \versiondate}	% footer even page left part 
}

\renewcommand{\chaptermark}[1]{\markboth{\thechapter.  #1}{}}
\renewcommand{\headrulewidth}{0pt}				% no header stripline
\renewcommand{\footrulewidth}{0pt} 				% no bottom stripline

\pagestyle{plain}
%---------------------------------------------------------------------------


% Title Page and Abstract
%---------------------------------------------------------------------------
%
% Project documentation template
% ===========================================================================
% This is part of the document "Project documentation template".
% Authors: brd3, kaa1
%

\begin{titlepage}


% BFH-Logo absolute placed at (28,12) on A4 and picture (16:9 or 15cm x 8.5cm)
% Actually not a realy satisfactory solution but working.
%---------------------------------------------------------------------------
\setlength{\unitlength}{1mm}
\begin{textblock}{20}[0,0](28,12)
	\includegraphics[scale=1.0]{../img/BFH_Logo_B.png}
\end{textblock}

% Institution / titel / subtitel / authors / experts:
%---------------------------------------------------------------------------
\begin{flushleft}

\vspace*{21mm}

\fontsize{26pt}{40pt}\selectfont 
\heading				\\							% Read heading from file leader/title.tex
\vspace{2mm}

\fontsize{16pt}{24pt}\selectfont\vspace{0.3em}
Hosed based intrusion detection system without hashing			\\				% Insert subheading
\vspace{5mm}

\fontsize{10pt}{12pt}\selectfont
\textbf{Bachelor thesis} \\		% Insert text
\vspace{7mm}

% Abstract (eingeben):
%---------------------------------------------------------------------------
\begin{textblock}{150}(28,100)
\fontsize{10pt}{12pt}\selectfont
In this thesis I show how a host based intrusion detection system can be built for scalability. More specifically I show how the sleuth kit can be used to create an intrusion detection system that does not rely on hashing, but on \gls{fs} attributes. This way it gains speed which enables us to take the risk not to calculate hashes.
\end{textblock}

\begin{textblock}{150}(28,225)
\fontsize{10pt}{17pt}\selectfont
\begin{tabbing}
xxxxxxxxxxxxxxx\=xxxxxxxxxxxxxxxxxxxxxxxxxxxxxxxxxxxxxxxxxxxxxxx \kill
Degree course:	\> Engineering and Information Technology	\\		% insert name of degree course
Author:		\> 		Julian Stampfli\\					% insert names
Tutor:	\> Dr.~Bruce Nikkel		\\							% insert names
Expert:		\> Armin Blum				\\							% insert names
Date:			\> \versiondate					\\							% read from file leader/version.tex
\end{tabbing}

\end{textblock}
\end{flushleft}

\begin{textblock}{150}(28,280)
\noindent 
\color{bfhgrey}\fontsize{9pt}{10pt}\selectfont
Berner Fachhochschule | Haute \'ecole sp\'ecialis\'ee bernoise | Bern University of Applied Sciences
\color{black}\selectfont
\end{textblock}


\end{titlepage}

%
% ===========================================================================
% EOF
%
		% activate for frontpage without picture
%\include{leader/frontpage_with_picture}		% activate for frontpage with picture
\cleardoubleemptypage
\setcounter{page}{1}
\cleardoublepage
\phantomsection 
\addcontentsline{toc}{chapter}{Abstract}
\chapter*{Abstract}
\label{chap:abstract}

Many tools exist to help a company to protect itself against cyber attacks. One type of such a tool is called an intrusion detection system (IDS). Those are created to detect attacks that somehow got through other measures and infected one or many hosts in the network. One type of IDS is called network-based intrusion detection system (NIDS). They operate on a network level and analyze the incoming and outgoing traffic for anomalies. Those anomalies usually signal an intrusion. When they find such an intrusion, they usually generate an alert for a system administrator or security professional to analyze. 

There is also another type of IDS called host-based intrusion detection system. They operate directly on the host and try to find attacks there. They are more effective at finding intrusions that are dormant and don't do anything for some time. They mostly operate on the file system and sometimes go beyond that. HIDS have been quite effective at finding intrusions on file basis in the past by creating cryptographic hashes of hashes and comparing them to previous executions. However, as file sizes have been growing, they began to struggle to execute within a short time frame. Calculating a hash is seen as the only reliable way to find changes to the file system, and with more data, it is taking increasingly long to calculate them. The situation has grown out of proportions because the time to scan now takes so long, that the intrusion detection system can't reliably find intrusions within a useful timespan.

In my thesis, I try to show a different solution to this problem. I created a host-based intrusion detection system that works at a file basis but does not calculate any hash. Instead, it finds intrusions by evaluating the file system attributes like modification time and permissions. This approach is risk-based because it is less reliable, but by increasing the speed, the host can be scanned multiple times more often than if hashes get calculated. I am using an open source forensic investigation tool called the sleuth kit (TSK). It offers much functionality for file system analysis and works on most operating systems. With this tool, I can extract the file system attributes reliably and fast, without touching the files themselves.

There is another advantage that I hope to give with my system. Forensic investigators usually struggle to reliably create a timeline of what happened on a file system after an intrusion. This timeline is essential because it can lead them to find out what exactly happened and often can make future intrusions harder. Here I want to help as well. Other than the other HIDS, my system stores all the executions. This way, an investigator can look at this history and sees how the attack started. One nice side-effect of that is that my system is very flexible. After changing something on the host, the system automatically adjusts.

With my tool, system administrators and forensic investigators have another option to tackle intrusion detection. Taking the risk-based approach can lead to many fast detections that otherwise would not be detected in time. Additionally, investigators have more data at their disposal to protect from future attacks.
\cleardoubleemptypage
\phantomsection 
\addcontentsline{toc}{chapter}{Acknowledgements}
\chapter*{Acknowledgements}
\label{chap:acknowledgements}

Firstly, I want to tank my tutor Bruce Nikkel and for his outstanding guidance and ongoing support during this project.

Next, I want to thank Amir Blum for supervising this work as my expert and for giving valuable feedback at the start of the project.

Further, I want to thank the \gls{opensource} community behind \gls{tsk} and \gls{pytsk} without whom this project would not have been possible. Especially Joachim Metz who is very active in the \gls{pytsk} community and helped me find the solution to one of my issues.

\cleardoubleemptypage
%---------------------------------------------------------------------------

% Table of contents
%---------------------------------------------------------------------------
\tableofcontents
\cleardoublepage
%---------------------------------------------------------------------------

% Main part:
%---------------------------------------------------------------------------
\pagenumbering{arabic}


\chapter{Introduction}

An \gls{ids} is used to protect a computer from \gls{malware} attacks. It does so by tracking and evaluating activity form one or many hosts. The \gls{ids} then tries to find \glspl{anomaly} usually by comparing the activity to some type of configuration which often contains some form of blacklist or whitelist. The \glspl{anomaly} are then alerted or logged to some framework. The process of intrusion detection is generally split into a host based part with \gls{hids} and a network based part with \gls{nids}. \gls{nids} are used to detect unusual behaviour of the network. Examples include communication from hosts which usually don't comunicate. \gls{hids} on the other hand are used to detect \glspl{anomaly} on a host. This is done by detecting changes in the way processes are used or when the \gls{fs} is changed. Both types of \gls{ids} have different advantages and they should be used in conjunction for best results. \cite{needed}

This Thesis is about writing of a \gls{hids} that operates on the \gls{fs}. As already mentioned a \gls{hids} operates on the host machine and detects \glspl{anomaly} by comparing resources available on the host. For detecting changes to the \gls{fs} a \gls{hids} usually generates \glspl{hash} and compares them to previous calculations. If the \gls{hash} changes, then the file has been altered. If this alteration is detected as an \gls{anomaly} it will be alerted. This approach has one weakness. The caclulation of a \gls{hash} takes time. Additionally to the time needed to actually calculate the hash, this approach needs to read the whole content of all relevant files, which takes up some more time. \cite{hash:slow, hash:speed} This is historically not relevant as it is efficient enough that the entire \gls{fs} could be hashed in a small amount of time. But storage media grew and with it the amount of data on a server. \cite{bruce:imaging} With that the calculation of \glspl{hash} needs more time and traditional \gls{hids} can't scan big systems within a valuable amount of time anymore. Finding anomalies based on \gls{metadata} is faster, because the files don't need to be parsed and the size of them doesn't matter. Thus, such a \gls{hids} can be run frequently and with that \glspl{intrusion} can be detected faster, which is essential for protection. \cite{inode}

The \gls{hids} written in this thesis tries to gain another advantage. \gls{hids} are used to protect against \glspl{intrusion}. They don't offer much support in \gls{investigation} once an \gls{intrusion} happend. Those \gls{investigation} capabilities are built in this implementation. 

\section{Drawbacks}

This implementation doesn't calculate \glspl{hash}, changes within a file can thus be hidden from the \gls{hids} if the intrusion adjusts all the \gls{metadata} of the changed files. For further information on this attack and how to mitigate it please see section \ref{sec:attack_scenarios}.


\chapter{Technical Background}

In this chapter I provide some technical introduction to relevant topics. Further information is avaliable in the linked resources.

\section{Definitions}

\subsection{Filesystem}
\label{sec:fs}

\gls{storagemedia} takes many forms. There are the traditional \gls{hdd}, there are newer \gls{ssd}. Both are built based on different assumptions and using different technologies. There are more \gls{storagemedia} like \gls{cd}, \gls{dvd}, \gls{usb} flash drives and more. \gls{storagemedia} operates on blocks of data. One such block is typicaly 512 bytes or 4 \gls{kb}. \cite{bruce:imaging}

An operating system typically needs to store files of data. This is where \glspl{fs} come in. They create a layer of abstraction for the \gls{storagemedia}. A \gls{fs} uses the blocks of the \gls{storagemedia} and stores the files in a data structure called inode. \cite{inode} The \gls{fs} also keeps track of \gls{metadata} like creation time, accessed time, permissions, etc. Which \gls{metadata} is stored depends on the \gls{fs}. \cite{bruce:imaging}

\subsection{Cryptographic Hashing Function}
\label{sec:hashing}

A hash function is generally a function which takes an input of unlimited size and generates an output of a fixed size. Hash functions are used widely in programming and databases to easily access certain data within a datastructure. By design, hash functions have collisions, meaning that multiple inputs generate the same output. This must be true if you consider the unlimited inputsize and limited output size. If there are more inputs than outputs there must be at least one value that is assigned to multiple inputs. For data storage and other usecases this is not a huge problem because collisions can be handled and if the hash function has a good distribution collisions are unlikely. In many systems a hashing function with weak collision resistance is deliberately chosen because it is faster to execute. \cite{hash:noncrypto, hash:slow}

In a cryptographic context this trade off can not be taken. There are two big factors that play into why not. Firstly in cryptography hashes are often used as an assurance that the content of some data has not changed. If collisions are easy to find, the data can be altered in ways that result in the same hash, meaning that the hash no longer fullfils the usecase. Additionally, the potential gain can be big. If a collision can be provoked, even if challenging, data can be changed again. This data could be a legal document or a bank transaction, neither of which we want to change. For those reasons a cryptographic hash function needs to be highly collision resistant. The drawback of collision resistant algorithms over easier hashing functions is operations needed for it, while modern hashing functions are performant and secure, they still take some time for a lot of data. \cite{crypto}

One \gls{hash} is called \gls{sha256}. \gls{sha256} is a standard published by the \gls{nist}. It creates a 256 bit output and has not yet been successfully atacked. \cite{sha} This hashing algorithm is used to make sure that the configuration of this implementation has not changed between multiple runs.


\subsection{HIDS}
\label{sec:def:hids}

A \gls{hids} works by detecting changes on the local host. It does that by looking at files, processes, configuration, logs or other indicators. In this thesis I will only focus on files. It is important to note that the other sources are an important and very valuable source of information. Any of those sources might have \gls{ioc}. Especially running processes and the configuration can hold important data. However, this implementation only covers file based information.

In such a system that finds intrusions via the \gls{fs} most information comes from unexpected changes on the \gls{fs}. Most hosts will not have any changes on the \gls{fs} except for patches. Thus, the easiest approach is to compare the current state of the \gls{fs} to previous ones. If it has changed significantly or in an unexpected way, something foul might be happening. The negative side effect of this approach is the false positives, that come from legitimately changing the host. Those legitimate changes could be new versions of the webpage that is running, new updated configuration or updated keypairs. However, those changes come scheduled and the alerts can then be quickly checked and acknowledged.

A good \gls{hids} should be able to handle such valid changes without much change. Additionally, it should be able to find intrusions reliably and in a timely fashion. Maybe it is to late if the \gls{hids} finds an intrusion a week after it infected the system. To detect changes on the files reliably current \gls{hids} calculate a hash of the files and compares that hash to previous runs. If a \gls{hash} is used, each file will always generate an unique hash which can not be faked. The main drawback of using good \glspl{hash} is that they take a longer time. Some implementations thus use weaker implementations of \gls{hash} or even \gls{nonhash}. The obvious drawback of this approach is that a collision can be generated and such a file can be altered or replaced without the \gls{hids} noticing. 

In the following sections I will present two of the most used \gls{hids}, Tripwire and Aide. They built the main competition of the integration developed in this thesis.

\subsubsection{Tripwire}
\label{sec:tripwire}

In 1992 the first \gls{hids} named tripwire was created and publicly released as a free tool. In 1997 the creator of tripwire then created the company Tripwire Inc. and bought the naming right for Tripwire. The free version was monetized and they released new versions of Tripwire. \cite{Tripwire:Impl,Tripwire:company} Now they mostly market to enterprise and industrial customer and have more products than only one for file integrity. \cite{tripwire}

As it is a comercial product it is hard to see how they work in detail since most of the information is only available to paying customer. 

\subsubsection{AIDE}
\label{sec:aide}

Aide is an \gls{opensource} alternative to tripwire. It was created after tripwire went comercial. \cite{aide:totherescue, aide:github}

When aide is first run, it generates a database as a reference. This database contains all the information for each file that is within a path of interest. Depending on the config it will contain more or less information, including \glspl{hash}. Each subsequent run will then compare the files found to this initially created database. The database needs to be updated manually if needed. It generates a log of all the changes and distributes that per email or similar if configured. Both the configuration and the database are usually stored on the file system that is scanned. \cite{aide, aide:doc}

Aide can compare multiple \glspl{hash} and \glspl{nonhash} and some \gls{metadata}. It runs on many modern unix system \cite{aide} and uses gpg keys to sign their releases. It has a strong comunity behind it, but only offers a \gls{cli} and has no fancy user interface and is written in C. \cite{aide:github} It also contains an extensive configuration covered in section \ref{sec:aide:config}.


\subsection{NIDS}
\label{sec:def:nids}

Compared to a \gls{hids}, a \gls{nids} can be used to detect intrusions over multiple hosts. \gls{nids} are used to analyze anomalies on the network. An \gls{ioc}	might be an unexpected session from a server which usually doesn't communicate with the internet. Also unusually large or frequent trafic might be suspicious. Certain intrusions can be tracked by analyzing the content of the packages. Some intrusions are easier to detect on the network. Especially such ones that extract a lot of data or are very aggressive at spreading within a network. The main advantage here is that multiple sources can be combined on the network level. Another advantage this approach has, is that current \gls{malware} almost always contains some kind of network communication. \cite{Malware:Behaviour,nids}

A \gls{nids} doesn't only have advantages though. Certain kinds of attacks can only be detected on the network with much difficulty. For a corrupted file upload can not really be detected by only checking the network. It is imperative that both types of \gls{ids} are used in conjunction to detect attacks. 

In this thesis \gls{nids} are not in the center. The implementation will not look at network traffic or mimik other \gls{nids} functionality. However, as mentioned above, I highly encourage everyone to use an appropriate \gls{nids} to improve the chance to find an intrusion. 

\subsection{The Sleuth Kit}
\label{sec:tsk}

\gls{tsk} is an \gls{opensource} toolkit used to investigate disk images. It is based on the coroner's toolkit \cite{tct} and contains multiple command line interfaces and an \gls{api} for various purposes. \cite{tsk, tsk:about} It is mainly written in C, runs on Linux, OsX, Windows and more and can be used to analyze many different \glspl{fs}. It is heavily used in forensic investigations to find deleted or corrupted files and for other information gathering on a disk image.

\subsubsection{fls}
\label{sec:fls}

One of those command line tools is fls. It can be used to access directories, files and the attributes of each. With it the directories can be displayed recursively and for each file the attributes can be printed to the console. \cite{tsk:fls} This tool plays a central part in this thesis as it is used to recover all the files and attributes to detect intrusions. On how exactly it is used please refer to section. \ref{sec:Scanner}

\subsection{Python}
\label{sec:python}

Python was the programming language of choice for this product. Python offers many advantages over other languages. Some of those are:

\begin{itemize}
	\item Platform independant
	\item Good community support in forensic community
	\item Active library for \gls{tsk}
	\item Small overhead
	\item Easy for prototyping
	\item Easy to read
\end{itemize}

This decision was not made very lightly. Other programming languages were considered. For more information on those refer to section \ref{sec:decisions:language}.

\subsubsection{pytsk3}
\label{sec:pytsk3}

Pytsk3 is the aforementioned library that creates bindings for python to the \gls{api} of \gls{tsk}. As \gls{tsk} this library is \gls{opensource} and is still active. It is hosted on \gls{github} and offers most of the functionality of the \gls{tsk} \gls{api}. Pytsk3 is used extensively in the scanner part of the implementation. For further information refer to section \ref{sec:Scanner}.

\section{Host based intrusion detection system}
\label{sec:hids}

The main principle behind a \gls{hids} is the detection of changed files. As already discussed in section \ref{sec:def:hids} most tools rely on the calculation of hashes. This is generally a good approach since changes can be found very reliably, however, as already mentioned, it can drastically hinder the performance of the \gls{hids}. Sadly the actual performance lost can not be clearly stated, as it heavily depends on what hardware is in use. But considering the computational overhead of calculating a \gls{hash} it is clear that the time it takes grows with bigger \glspl{fs}. As \gls{storagemedia} has grown from a \gls{mb} to \gls{tb} so has the requirement to store more data. Creating hashes over such big systems is not viable as it can take a long time to create a hash of big amounts of data. \cite{hash:slow, hash:veryslow, hash:speed}

\subsection{Proposed solution for time issue}

In this thesis I propose a different approach. Forget the hashing, and the content of the file. The file attributes suffice to catch an intrusion. The main advantage of not calculating the hashes is the improved speed. The \gls{hids} can thus run way more often. If a conventional \gls{hids} might take several days to complete a run on a \gls{fs} with multiple \gls{tb} of data the proposed approach would maybe take some minutes. It could then be run hourly and find new and changed files within an hour at most. It is possible that the \gls{hids} will miss some changes if the attacker can change all the \gls{metadata} before the system checks the same file again, but this is a risk that has to be taken to gain the opportunity to scan large \gls{fs}. It is also possible to scan highly critical sections of a system with a traditional \gls{hids} and the rest with the proposed solution. This way one has both advantages. The whole system can be checked in a timely fashion with the proposed solution and for certain smaller parts of the system a general \gls{hids} can detect changes by using strong \glspl{hash}.

This solution uses \gls{tsk} via \gls{pytsk} to extract the \gls{fs} \gls{metadata}. The main advantage that this gives is the interoperability with different \glspl{fs}. Additionally, by directly accessing the attributes from the \gls{fs}, no \gls{metadata} is actually changed. The files themselves are never touched. Furthermore, it can also be used to get the files and attributes of an image that has been extracted or of a virtual machine. 

\subsection{Investigation capabilities}
\label{sec:investigation:capabilities}
Another improvement upon existing \gls{hids} proposed is that investigation capabilities need to be built in from the start. Traditional \gls{hids} are good at finding intrusions and alerting them. However, they don't offer support for investigation of the incident. They don't have information bejond what was configured. This makes sense if a lot of the information they gather is through the calculation of \gls{hash}. However, if something was missed in the configuration an investigator can't use the output of those systems to gain further knowledge. They only have one output and nothing further.

This solution stores all the available \gls{metadata} for each scanned file for each run. This way an investigator can use this output to gain more information about how the attacker proceeded. He can look at the changes of permissions, modification of files, even if the alerts might have been ignored. It is also possible to generate a timeline out of this data to form a extended view on what happened when on the \gls{fs}. 

\subsection{Flexibility}

Aide works by comparing runs to one initial execution. This is practical as it will detect one intrusion multiple times. However, it will generate a lot of messages and users are then less likely to take them seriously. Additionally to that, after a legitimate change to the system, Aide will always generate alerts until the initial run is reset. This can lead to undetected intrusions a short time after an upgrade. This might also be the most important window for an intrusion because the update might have created a security risk. Thus, it is possible that an intruder can gain access shortly after an update which will be alerted by Aide, but ignored because of the other false positives.

The proposed system does not have this issue, at least not as strongly. As all the data gets collected for each run anyways, it makes sense to compare each run to the previous. This has the benefit of legitimate changes being adopted into the accepted one run after it has been finished. This results in overall less messages which increases the importance of each. If it is required that the system should always compare against one specific run, this could also easily be done. The system would need to be configured to not check against the latest, but against one specific run. To update this run only the configuration would then need to be changed. 

\section{Scope}

In this project I create a \gls{hids} wich uses \gls{tsk} to detect changes. It will cover the three main changes discussed in section \ref{sec:hids}. This service uses a \gls{sql} database to store the executed runs. It includes an user documentation and this thesis documentation. 

Out of scope are the creation of the timeline mentioned in section \ref{sec:investigation:capabilities}. Also out of scope are extensive alerting functionality and extensive example configurations for commonly used operating systems and tools. Furthermore, any big data analysis of the runs, while very interesting, are also out of scope.


\include{chapter/3results}

\chapter{Results}

In this chapter, I will talk more about the implementation details of the \gls{fids}. I will also look into the security of this implementation and what measures can be implemented to improve it. 

\section{System Architecture}
\label{sec:Architecture}

\begin{figure}[ht]
	\includegraphics[width=8cm]{../img/Overview_FIDS.png}
	\centering
	\caption{System Architecture}
	\label{fig:systemArchitecture}
\end{figure}

As seen in figure \ref{fig:systemArchitecture}, the \gls{fids} contains two main components, the scanner and the investigator, a database and has connections to some libraries. It is split into scanner and investigator way to gain flexibility, mainly so that both components can be executed independently. This independence has the advantage that the scanner and the investigator don't need to be run on the same system. Also, it means that previous results of the investigator can be recreated by running the investigator on previously captured data. Additionally, if someone is only interested in the results of the scanner, the investigator doesn't need to be run. The scanner is explained in detail in section \ref{sec:Scanner} and the investigator in section \ref{sec:Investigator}.

Besides the scanner and the investigator which were produced in this thesis, there are five other components which build the environment of the \gls{hids}. There is the \gls{fs} which contains the data in which we are interested. As already mentioned, this could be a mounted device that is directly accessible through the operating system. It could also be run on the disk images of virtual machines or previously created disk images. To run it on a previously created disk image might be interesting if used on backup images if backups are made by creating disk images. 

Then there is the forensic component. This component is the combination of \gls{tsk} and \gls{pytsk}. The main purpose of this component is the abstraction of the \gls{fs} into python classes, which can then be used within the scanner. This abstraction is important to be compatible with multiple different \glspl{fs}. \gls{tsk} offers much different functionality; the \gls{fids} is only using the utility that is used for the fls command described in section \ref{sec:fls}. 

This leaves the configuration and the database. Both of which are also part of this thesis. The database is used to store the data of each run, and the configuration is used to change the behavior of the \gls{fids}. The configuration is explained in depth in section \ref{sec:Configuration} and the database in section \ref{sec:Database}.

There is another component that is not part of the \gls{hids} functionality. The \gls{fids} contains a component that is used to create a timeline. This component is special because it is designed for use by forensic investigators. It is explained in section \ref{sec:Timeliner}.

\subsection{Configuration}
\label{sec:Configuration}

The configuration file is provided in \gls{yaml}. \gls{yaml} is a human-friendly data serialization language with support for many programming languages. Its main advantage over other languages for configuration files is the readability and the ease to extend already existing configuration files. There were more reasons why \gls{yaml} is used documented in appendix section \ref{sec:decisions:config:language}.

\subsubsection{Database Configuration}

The configuration consists of three parts. The first part is the database configuration. The current implementation supports only \gls{sqlite}, as this is the most basic and easy to use format. Additionally, it doesn't require any connection to an external entity. For more information on why \gls{sqlite} was chosen see appendix section \ref{sec:decisions:dbms}. For the configuration, only the filename is required as seen in listing \ref{lst:cfg:sqlite}. This configuration defines where the data will be sent to and read from in the scanner and investigator parts, respectively. As both parts need access to the database, this part of the configuration is in a separate part. It is possible and already prepared to extend the system to use more \gls{dbms}. However, this is not part of the scope of this thesis.

\begin{lstlisting}[language=yaml, numbers=left, caption=SQLite Configuration, label=lst:cfg:sqlite]
sqlite:
	filename: fids_db3.db
\end{lstlisting}

\subsubsection{Scanner Configuration}

The second part of the configuration is the part that defines the scanner. An example config can be seen in listing \ref{lst:cfg:scanner}. It consists of one main key, which is named scan. If this config entry is missing, the scanner part of the \gls{hids} is not executed by default. Beneath this top key it contains the following subkeys:

\begin{description}
	\item [image\_path] Path to the \gls{fs} that is used for the scan.
	\item [scan\_paths] List of paths to scan. Paths are scanned recursively. "/" thus scans all available paths.
	\item [ignore\_paths] Paths that should not be scanned. Can be any directory. The recursion stops once this path is reached and does not continue downwards. Practical if certain directories are not interesting for intrusion detection.
\end{description}

\begin{lstlisting}[language=yaml, numbers=left, caption=Scanner Configuration, label=lst:cfg:scanner]
scan:
	image_path: /dev/nvme0n1p1
	scan_paths: 
		[
			"/",
			"/nonExisting",
		]
	ignore_paths: 
		[
			"/temp/"
		]
\end{lstlisting}

\subsubsection{Investigator Configuration}
\label{sec:conf:investigator}

The investigator configuration is similar to the scanner configuration in the way that it contains a top-level node called investigator. If this is missing the investigator part does not start. As can be seen in listing \ref{lst:cfg:investigator} it is a lot more complicated than the scanner configuration. 

\begin{description}
    \item [same\_config] This configuration specifies whether a changed config should result in an alert. Defaults to True.
    \item [validation\_run] This can define a run that is used for validation. If empty the second last one will automatically be used.
    \item [\nameref{sec:config:rules}] Rules are ways to create templates on how changed files should be found. \nameref{sec:config:rules} are explained more through below.
    \item [\nameref{sec:config:investigations}] Investigations define which paths and which files should be checked for intrusions. \nameref{sec:config:investigations} are explained more through below.
\end{description}

\begin{lstlisting}[language=yaml, numbers=left, caption=Investigator Configuration, label=lst:cfg:investigator]
investigator:
	same_config: True
	validation_run: 
	rules: 
		- name: php
		  rules: 
			- m
			- i
			- l
			- n
			- a
		  equal:
			- meta_creation_time
			- meta_size
		- name: logs
		  greater:
			- meta_size
		  equal:
			- meta_creation_time
	investigations:
		- paths:
			- '/etc'
		  fileregexwhitelist:
			- '*.php'
		  rules:
			- php
		- fileregexwhitelist: '/*'
		  fileregexblacklist: '*evilfile*'
		  rules:
			- logs
\end{lstlisting}

\subsubsection{Rules}
\label{sec:config:rules}

Rules represent templates that are later used in the investigations to find anomalies. Rules can be based upon other rules to extend them. Recursive rules are allowed, they simply extend each other. An example of the rule configuration can be seen in listing \ref{lst:cfg:investigator}, the fields are explained below.

\begin{description}
	\item [name] Each rule has a name. If two rules with the same name are defined, the behavior might be inconsistent. The name is used to extend rules.
	\item [rules] The rules which are used as a basis for this rule. They are referenced by name.
	\item [greater] The properties which are allowed to grow between the runs. Greater also includes equal.
	\item [equal] The properties that are supposed to stay equal during all runs.
\end{description}

Additionally, there are some predefined rules. The aide configuration heavily influenced which rules got predefined. The preconfigured rules are listed in listing \ref{lst:cfg:precon}.
\begin{lstlisting}[language=yaml, numbers=left, caption=Preconfigured Rules, label=lst:cfg:precon]
rules: 
	- name: p
	  equal:
		- meta_mode
	- name: ftype
	  equal:
		- meta_conten
	- name: i
	  equal:
		- meta_addr
	- name: l
	  equal:
		- meta_link
	- name: n
	  equal:
		- meta_nlink 
	- name: g
	  equal:
		- meta_gid
	- name: s
	  equal:
		- meta_size
	- name: m
	  equal:
		- meta_modification_time
		- meta_modification_time_nano
	- name: a
	  equal:
		- meta_access_time
		- meta_access_time_nano
	- name: c
	  equal:
		- meta_changed_time
		- meta_changed_time_nano
	- name: S
	  greater:
		- meta_size
\end{lstlisting}
	
Additional to those, there are some rules which do not directly check the \gls{fs} \gls{metadata}. There are three of them which change the behavior of the investigator. They must be configured on the investigation level. They are directly referenced by the investigator to specify if the related events should result in an alert or not. 

\begin{description}
    \item [file\_rename\_ok]    Usually a changed filename leads to an alert. This behavior is deactivated by setting this rule.
    \item [new\_files\_ok]    Usually a new file leads to an alert. This behavior is deactivated by setting this rule.
    \item [deleted\_files\_ok]    Usually a deleted file leads to an alert. This behavior is deactivated by setting this rule.
\end{description}

\subsubsection{Investigations}
\label{sec:config:investigations}

The investigation configuration defines the objects which are scanned for intrusion, as well as how they are scanned. They contain rules and some configuration on what should be scanned. An example of the investigation configuration can be seen in listing \ref{lst:cfg:investigator}, the fields are explained below.

\begin{description}
    \item [paths] For investigations different paths can be defined. This setting changes the behavior in such a way that only the specified path are investigated. Useful if certain paths need different observations.
    \item [rules] The rules which are used within this investigation is based on are defined here. They are referenced by name.
    \item [fileregexwhitelist] The \gls{regex} whitelist works similarly to the paths config. Only files are analyzed that match this \gls{regex}. It is parsed before the blacklist.
    \item [fileregexblacklist] The \gls{regex} blacklist is used to detect files based on their filename. If their path with filename results in a match with this \gls{regex} it is alerted. Especially useful for finding unexpected files in locations that have frequent changes.
\end{description}

\subsection{Database}
\label{sec:Database}

The database is another component that is shared between the scanner and the investigator. As already mentioned, multiple \gls{dbms} could be used in conjunction with the \gls{fids}. In this section, I do not focus on the \gls{dbms} used but rather use general examples only using \gls{sql}. The database consists of five relations. Firstly, I explain some reoccurring types, how they are stored, and what they represent after that, I explain the relations.

Some ideas are not specific to a relation. Most relations contain timestamps and \gls{uuid}. How they are stored can be seen below:

\begin{description}
	\item [Timestamps] To save space, timestamps are stored in the \gls{unixts} representation. 
	\item [\gls{uuid}] \gls{id} are stored as \gls{uuid} in their \gls{hex} representation. 
	\item [Enum] \gls{tsk} defines many enumerations. Most of those enums have a number representation. To save space, this representation is stored in the database. The enums and their representation can be viewed in the \gls{tsk} \gls{api} reference. \cite{tsk:file:header}
\end{description}

\subsubsection{FIDS\_RUN}

The run relation is relatively simple. It contains an \gls{id}. This \gls{id} is used in the other relations as well to create a link to the run. Besides that, it contains a \gls{sha256} hash of the configuration that it was used. It then contains a start and end timestamp which represent the times when it started and when it ended. A run that is not yet completed only contains the start timestamp. The relation definition is shown in listing \ref{lst:cfg:fids:run}

\begin{lstlisting}[language=sql, numbers=left, caption=Fids Run Table Definition, label=lst:cfg:fids:run]
CREATE TABLE FIDS_RUN(
	id varchar(32), 
	config_hash varchar(64), 
	start_time int, 
	finish_time int, 
	PRIMARY KEY(id)
);
\end{lstlisting}

\subsubsection{FIDS\_ERROR}

Each execution has the possibility of creating errors. Those errors are stored in this table. It is rather simple as well. Additionally to the run \gls{id} it contains an \gls{id} for the error as well. Next, it contains a description and a location where the error occurred. The table definition is shown in listing \ref{lst:cfg:fids:error}

\begin{lstlisting}[language=sql, numbers=left, caption=Fids Error Table Definition, label=lst:cfg:fids:error]
CREATE TABLE FIDS_ERROR(
	run_id varchar(32), 
	id varchar(32), 
	description text, 
	location varchar(255), 
	PRIMARY KEY(run_id, id)
);
\end{lstlisting}

\subsubsection{FIDS\_FILE}

The file relation contains most of the information. Has an \gls{id} additional to the \gls{id} of the run. Then it has the path in which the file is located and all the \gls{metadata} about the file. Most attributes start with `meta' or `name'. This naming is a reference to \gls{tsk} which has separate structs for meta and name information about each file. I kept the naming of \gls{tsk}. Thus more information about each attribute can be found on the \gls{api} reference of \gls{tsk}. There are also two indices, one on the inode address and one on the combination of path and file. Those are required to reduce the execution time of the investigator \cite{tsk:file:struct}. The table definition is shown in listing \ref{lst:cfg:fids:file}.

\begin{lstlisting}[language=sql, numbers=left, caption=Fids File Table Definition, label=lst:cfg:fids:file]
CREATE TABLE FIDS_FILE(
	run_id varchar(32),
	id varchar(32),
	path text, 
	meta_addr int,
	meta_access_time int,
	meta_access_time_nano int,
	meta_attr_state int,
	meta_content_len int,
	meta_content_ptr int,
	meta_creation_time int,
	meta_changed_time int,
	meta_creation_time_nano int,
	meta_changed_time_nano int,
	meta_flags int,
	meta_gid int,
	meta_link int,
	meta_mode int,
	meta_modification_time int,
	meta_modification_time_nano int,
	meta_nlink int,
	meta_seq int,
	meta_size int,
	meta_tag int,
	meta_type varchar(255),
	meta_uid int,
	name_flags int,
	name_meta_addr int,
	name_meta_seq int,
	name_name int,
	name_size int,
	name_par_addr int,
	name_par_seq int,
	name_short_name int,
	name_short_name_size int,
	name_tag int,
	name_type varchar(255),
	PRIMARY KEY (run_id, id)
);
CREATE INDEX inode ON FIDS_FILE(meta_addr);
CREATE INDEX fullpath ON FIDS_FILE(path, name_name);
\end{lstlisting}

\subsubsection{FIDS\_FILE\_ATTRIBUTE}

In \gls{tsk} each file can contain multiple attributes. Those attributes are stored in this table. It contains the \gls{id} of both run and file and an additional one for the attribute itself. The attributes contain flags, a name and a type. The flags and the type enums called `TSK\_FS\_ATTR\_FLAG\_ENUM' and `TSK\_FS\_ATTR\_TYPE\_ENUM'. More information available in the \gls{tsk} \gls{api} reference \cite{tsk:attr:struct,tsk:file:header}. The table definition is shown in listing \ref{lst:cfg:fids:file:attr}

\begin{lstlisting}[language=sql, numbers=left, caption=Fids File Attribute Table Definition, label=lst:cfg:fids:file:attr]
CREATE TABLE FIDS_FILE_ATTRIBUTE(
	run_id varchar(32),
	file_id varchar(32),
	id varchar(32),
	flags int,
	tsk_id int,
	name varchar(255),
	name_size int,
	at_type varchar(255), 
	PRIMARY KEY (run_id, file_id, id)
);
\end{lstlisting}

\subsubsection{FIDS\_FILE\_ATTRIBUTE\_RUN}

Attributes can contain multiple data runs. Those data runs are represented in this relation. It contains the \gls{id} from run, file and attribute and one for the data run. It then has a block address and a lenght. More information can again be found in the \gls{tsk} \gls{api} reference. \cite{tsk:attr:run:struct} The table definition is shown in listing \ref{lst:cfg:fids:file:attr:run}

\begin{lstlisting}[language=sql, numbers=left, caption=Fids File Attribute Run Table Definition, label=lst:cfg:fids:file:attr:run]
CREATE TABLE FIDS_FILE_ATTRIBUTE_RUN(
	run_id varchar(32),
	file_id varchar(32),
	attribute_id varchar(32), 
	id varchar(32), 
	block_addr int, 
	length int, 
	PRIMARY KEY(run_id, file_id, attribute_id, id) 
);
\end{lstlisting}

\subsection{Shared}

There are other shared parts in the source code. Mainly the model of the system. It contains multiple classes that define the types with one type for each relation. They can parse \gls{tsk} objects and database rows. This way, both the scanner and the investigator can work with the same classes. Except for parsing, they don't have any functionality.


\section{Application Flow}
\label{sec:flow}

The main application flow consists of first reading the config file. If no config file path is passed the default path is used. It then starts the scanner if a scan configuration is found. After the scanner is completed it starts the investigator if the detection config is found. It works if any of the two configs are found. If none is found then the system will simply do nothing. 


\subsection{Scanner}
\label{sec:Scanner}

The scanner component is the first specific component. It is executed when the system is run and a scan config is available. It is responsible for getting all the information from the \gls{fs} for the configured paths. It has multiple stages.

\begin{enumerate}
	\item Initialization
	\item Scan
	\item Error Logging
	\item Storing
	\item Error Logging
	\item Finalizing
\end{enumerate}

In the initialization phase the scanner creates a run object. This object is already saved to the database. This way user know just by looking at the database if there are any runs still running. This also creates a first opportunity to look for inconsistencies. If there are a lot of started runs, something suspicious might be going on. It also creates the hash of the configuration and already saves it to the database.

The scan phase has more steps to it. First it creates the \gls{pytsk} object called `img\_info'. This is done by passing the path to the disk image to \gls{pytsk}. Then the actually important `fs\_info' object is created by passing the img info object. This way we have access to the \gls{fs}. The scan actually starts by calling `open\_directory\_rec' for each scan path. This function keeps track of all diretories it already traversed into as to avoid circular loops and unnecessary steps. Then it checks if the path is in the ignored paths. It then iterates over all objects in the directory. 

For all entries it checks if it is a valid entry with the required attributes to continue. Afterwards it checks if it is a directory. For all the directories the function first checks if the directory has already been visited and if not it calls itself with the new directory as the parameter. If the entry is a file, it is parsed into a python object and then stored into a local list of found files. Any errors that occur are saved by creating an error object that is stored into a list of errors.

In the error logging phase all errors that have occured in the scan process are stored in the database. This is done by iterating over all the errors and saving them one by one. The database is then commited and even if the files can't be saved for some reason, the errors will be persisted.

After the first error logging phase is the file saving phase. Here the files are saved again by iterating over them and saving one by one. Should any additional error occur while saving the files, they are again added to a list of errors. 

There is a second error logging phase. In this phase the errors that occured while saving the files get stored into the database. The functionality is the same as for the first error logging phase.

In the finalizing phase the run object gets updated and the endtime added. It is then also updated on the database. At this point the scanner is finished. It has written all the files from the \gls{fs} to the database. Also all errors that occured during this process are logged. The run object on the database will have a valid start and end timestamp. 


\subsection{Investigator}
\label{sec:Investigator}

TODO: finish investigator logic first. Could document current logic, but is not useful as it is not fast enough. 



\include{chapter/33security}

\chapter{Discussion}
\label{sec:Discussion}

The designed and implemented \gls{hids} works differently than previous \gls{hids}. Many will argue that because it does not generate \glspl{hash} it is insecure. I argue against that, because finding intrusions always takes risks. If a conventional \gls{hids} finds more intrusions but takes a long time to scan the whole system, it is not useful. Finding an intrusion days or even weeks after it happened might not be interesting. The attacker then has a long time to either hide, move on or extract all the information he is interested in. Finding intrusions in a timely fashion is very important and this is where my implementation shines. Additionally, even if it didn't find the intrusion, the database can be very helpful for forensic investigators to find out what the attacker did. This might help finding similar attacks in the future by adapting the configuration or by chaning other components. The argument this system makes is not even to be the one and only. In its current form this does not make sense. I hevily advise people to use a \gls{nids} or a \gls{hids} with other focusses. I also advise people to use a \gls{hids} on file basis which uses \glspl{hash} for highly critical parts of the system. However, if only a hash based \gls{hids} is used, then many attacks will be found to late, or not at all. Another benefit that this implementation has is that it runs on any system. \gls{tsk} runs on Linux, OsX and Windows, so does python. By using those components, this system should work on any of those systems as well.

During my work I realized the system and I tested it with modified data. However, the system has not yet been used in a productive environment and has not yet detected an intrusion. This means that it can not be said without doubt if it would work. The main reason why it has not been tested is that the time ran out. Besides that there were ethical and judical conserns of creating a host that is easily exploitable just so that it can be attacked. This would lead to criminals gaining access to a host to do their work which is not in my interest. Even if my system would find their intrusions, it is still possible that they can abuse the host for some time. Additionally, it is possible that they would use an attack which my system can not detect. This would mean that they have access to the host for a prolonged time. The system would needs to be tested in a live environment where attacks naturally happen. Sadly, I did not have access to such a system. 

\section{Future Work}
\label{sec:future:work}

It would make sense to extend this \gls{hids} to extend past the \gls{fs}. It would be great if the scanner can also scan the processes and network connections. Not only would this data be important to finding intrusions by the system, but it could also give investigators much more information. Information which they are currently mostly missing.

The system should also be extensively tested. For this it needs to run for a prolonged time in a productive system. The output and the found intrusions would then need to be compared to other system to gain important information about how fast and how many intrusions are found using this system. 

One other path that could be improved on would be the investigator. Currently it operates only on a configuration which someone must write. It would be great to autonomously analyse the scanner output and find anomalies on them. This could be done if a lot of data has already been collected on many different types of hosts. The types could then be grouped and an algorithm could detect similar patterns to find simmilar intrusions. 

Generally, the field of finding intrusions has a lot of opportunities for research. This system can be one part of an extensive system that checks for intrusions. Especially since disk sizes and data usage is still growing it is important to have such a system that can find intrusions fast.

It would also be interesting to add the output of the scanner to a super timeline. It would help seeing what happened on the host at any point in time. 


\chapter{Conclusion}
\label{sec:Conclusion}

In this thesis the goal was to create a \gls{hids} which finds intrusions in big \gls{fs} fast and helps with forensic investigation. Such a system was built. I am using \gls{tsk} to only check \gls{fs} \gls{metadata} to find intrusions. This way the system is extremely fast compared to systems that use \glspl{hash}. My system thus takes a risk based approach to finding intrusions. It will not find them as reliably as other systems, but it will find them faster. 


%---------------------------------------------------------------------------

% declaration of authorship
%---------------------------------------------------------------------------
\cleardoublepage
\phantomsection 
\addcontentsline{toc}{chapter}{Declaration of authorship}
\chapter*{Declaration of primary authorship}
\label{chap:declaration_authorship}

\vspace*{10mm} 

I hereby confirm that I have written this thesis independently and without using other sources and resources than those specified in the bibliography. All text passages which were not written by me are marked as quotations and provided with the exact indication of its origin. 

\vspace{15mm}

\begin{tabbing}
xxxxxxxxxxxxxxxxxxxxxxxxxxxxxx\=xxxxxxxxxxxxxxxxxxxxxxxxxxxxxx\=xxxxxxxxxxxxxxxxxxxxxxxxxxxxxx\kill
Place, Date:		\> Biel, \versiondate \\ \\ 
Last Name, First Name:	\> Julian Stampfli \\ \\ \\ \\ 
Signature:	\> ......................................\\
\end{tabbing}

%---------------------------------------------------------------------------
[1]CITE NEEDED PLEASE LOOK FOR SOMETHING,https://startpage.com/, [Accessed; 2019-06-10].[2] L. Latinov,MD5, SHA-1, SHA-256 and SHA-512 speed performance,https://automationrhapsody.com/md5-sha-1-sha-256-sha-512-speed-performance/, [Accessed; 2019-06-10].[3] W. Nevelsteen and B. Preneel, “Software performance of universal hash functions,” inAdvances inCryptology — EUROCRYPT ’99, J. Stern, Ed., Berlin, Heidelberg: Springer Berlin Heidelberg, 1999,pp. 24–41,isbn: 978-3-540-48910-8.[4] B. Nikkel,Practical Forensic Imaging. No Starch Press, 2016,isbn: 1593277938.[5]Detailed Understanding Of Linux Inodes With Example,https://linoxide.com/linux-command/linux-inode/, [Accessed; 2019-06-10].[6]Cryptographic and Non-Cryptographic Hash Functions,https://dadario.com.br/cryptographic-and-non-cryptographic-hash-functions/, [Accessed; 2019-06-10].[7] C. Paar and J. Pelzl,Understanding Cryptography: A Textbook for Students and Practitioners, 1st.Springer Publishing Company, Incorporated, 2009,isbn: 3642041000, 9783642041006.[8] National Institute of Standards and Technology (NIST),FIPS PUB 180-4: Secure Hash Standard (SHS).Aug. 2015. [Online]. Available:https://nvlpubs.nist.gov/nistpubs/FIPS/NIST.FIPS.180-4.pdf.[9] G. H. Kim and E. H. Spafford, “The design and implementation of tripwire: A file system integritychecker,” inProceedings of the 2Nd ACM Conference on Computer and Communications Security,ser. CCS ’94, Fairfax, Virginia, USA: ACM, 1994, pp. 18–29,isbn: 0-89791-732-4.doi:10.1145/191177.191183. [Online]. Available:http://doi.acm.org/10.1145/191177.191183.[10] ,https://www.tripwire.com/company/, [Accessed; 2019-06-10].[11]Tripwire,https://www.tripwire.com/, [Accessed; 2019-06-10].[12] A. Messenger,AIDE to the Rescue – An Open Source Security Tool,http://www.drdobbs.com/aide-to-the-rescue-an-open-source-sec/199101554, [Accessed; 2019-06-10].[13]aide source code,https://github.com/aide/aide, [Accessed; 2019-06-10].[14]AIDE,https://aide.github.io/, [Accessed; 2019-06-10].[15]The AIDE manual,https://aide.github.io/doc/, [Accessed; 2019-06-10].[16] U. Bayer, I. Habibi, D. Balzarotti, E. Kirda, and C. Kruegel, “A view on current malware behaviors,”inProceedings of the 2Nd USENIX Conference on Large-scale Exploits and Emergent Threats: Botnets,Spyware, Worms, and More, ser. LEET’09, Boston, MA: USENIX Association, 2009. [Online]. Available:http://static.usenix.org/events/leet09/tech/full_papers/bayer/bayer.pdf.[17] G. Vigna and R. A. Kemmerer, “Netstat: A network-based intrusion detection approach,” inProceedings14th Annual Computer Security Applications Conference (Cat. No.98EX217), Dec. 1998, pp. 25–34.doi:10.1109/CSAC.1998.738566.[18]The Coroner’s Toolkit (TCT),http://www.porcupine.org/forensics/tct.html, [Accessed; 2019-06-10].[19]Overview,https://www.sleuthkit.org/sleuthkit/, [Accessed; 2019-06-10].[20]About,https://www.sleuthkit.org/about.php, [Accessed; 2019-06-10].[21]FLS,http://www.sleuthkit.org/sleuthkit/man/fls.html, [Accessed; 2019-06-10].[22]How to Calculate a Hash of a file that is 1 Terabyte and over?https : / / stackoverflow . com /questions/22724744/how-to-calculate-a-hash-of-a-file-that-is-1-terabyte-and-over,[Accessed; 2019-06-10].Alternative scalable HIDS with investigation capability, Version 1.0.1, June 3, 201931

% Glossary
%---------------------------------------------------------------------------
\cleardoublepage
\phantomsection 
\addcontentsline{toc}{chapter}{Glossay}
%\renewcommand{\glossaryname}{Glossay}
\printglossary[type=acronym]
\printglossary
%---------------------------------------------------------------------------

% Bibliography
%---------------------------------------------------------------------------
\cleardoublepage
\phantomsection 
\printbibliography[heading=bibintoc]

%---------------------------------------------------------------------------

% Listings
%---------------------------------------------------------------------------
\cleardoublepage
\phantomsection 
\addcontentsline{toc}{chapter}{List of figures}
\listoffigures
\cleardoublepage
\phantomsection 
\addcontentsline{toc}{chapter}{List fo tables}
\listoftables
\cleardoublepage
\phantomsection 
\addcontentsline{toc}{chapter}{List fo code listings}
\lstlistoflistings

%---------------------------------------------------------------------------

\appendix
\settocdepth{section}
% !TEX root = ../thesis.tex

\chapter*{APPENDICES}
\addcontentsline{toc}{chapter}{APPENDICES}
\settocdepth{section}

\begingroup\let\clearpage\relax
\chapter{Project Management}
\endgroup

\section{Goal}
\label{apdx-sec:goal}
Before the start of the project the following main goal was defined:

Building of an \gls{hids} that detects unauthorized or unusual behaviour on the file system. Compared to traditional \gls{hids} file system integrity checking, it should scale with a lot of data and have the possibility to be used for investigation (retain historic data) built in from the start.

\subsection{Sub goals}

From this primary goal, the following sub goals were defined. 

\subsubsection{Scanning}
The system is capable of scanning the file system for certain properties. The search is done by leveraging the sleuthkit tools. Thus the system is capable of interpreting the results from sleuthkit. It will further analyze them and decide on what to do with the results. Especially importance is given to the finding of differences.

\subsubsection{Recording}
The system records all findings. Including new, changed and deleted files in comparison to an earlier point in time. This recording enables the use of investigation as the evolution of the data can be viewed at any time. This data can also be used for machine learning algorithms to detect anomalies that are out of the scope of this thesis. 

\subsubsection{Evaluation}
The system is capable of evaluating the results by applying predefined rules. Those rules can be adjusted by configuring the system.

It is thinkable that the system analyzes the recordings and makes decisions based on the historical behavior of the specific host and behavior from different similar hosts. This approach is not part of this thesis as it requires much historical data that is not present at the time of this thesis. 

\subsubsection{Alerting}
The system is capable of being run continuously. This capability enables it to find anomalies automatically. The system can report those anomalies by creating alerts. It allows configuration of these alerts.

\subsubsection{Scaling}
The run of the system on a big file system completes in an appropriate amount of time. This speed allows the finding of anomalies that appeared recently. Additionally, it allows the storing of more states of the system which results in a her probability of capturing short-lived anomalies for future investigations. 

\section{Workpackages}

From those goals the workpackages in table \ref{tab:workpackages} were defined. For a better overview they are assigned to categories. The categories are Architecture, Implementation, Validation and Administrative. Architecture is about defining how the system will look like and how it should work. Implementation is the effective implementation work for getting the system to run, this includes configuration and coding. Validation is about testing of the system. Administrative is everything that deals with project management and other workpackages that don't directly influence the system but need to be done.

The ID is a combination of the first letter of the category and a unique index. Administrative is shortened to D due to the conflict with Architecture.

The priority is a value of high, medium and low. 

The workpackages are chrononically ordered. Meaning they should be worked on in approximately the order that they are given. 

\begin{table}[!ht]
  \begin{center}
    \caption{Workpackages}
    \label{tab:workpackages}
    \begin{tabular}{c|l|c|l}
      \textbf{ID} & \textbf{Short description} & \textbf{Prio} & \textbf{Category} \\
      \hline
		D00 & Setting up \LaTeX -document & h & Admin. \\
		D01 & Define workpackages and set deadlines & h & Admin. \\
		A00 & Research other \gls{hids} and the sleuthkit tools & h & Archit. \\
		A01 & Decide on a Programming Language & h & Archit. \\
		I00 & Setup the developer environment & h & Impl. \\
		I01 & Add the ability to scan the whole system using sleuthkit & h & Impl. \\
		A02 & Decide which database connectors should be used & h & Archit. \\
		I02 & Add one database connector & h & Impl. \\
		I03 & Implement a recording functionality & h & Impl. \\
		A03 & Decide how the rules should be defined & h & Archit. \\
		I04 & Add template rules and ability to parse them & h & Impl. \\
		I05 & Add functionality to parse output according to rules & h & Impl. \\
		V00 & Verify that the system runs on a big file system & h & Valid. \\
		I06 & Add functionality of repeated scans & m & Impl. \\
		A04 & Define which alerting methods make sense & m & Archit. \\ 
		I07 & Add alerting functionality using one method & m & Impl. \\
		V01 & Verify the functionality of the software by changing the system & h & Valid. \\
		V02 & Verify the functionality of the software by running it on an infected system & m & Valid. \\
		V03 & Verify the alerting of the software by running it on an infectable system & m & Valid. \\
		I08 & Add multiple database connectors to different systems & m & Impl. \\
		I09 & Add multiple alerting methods & m & Impl. \\
		A05 & Define how to protect system and configuration from tampering & l & Archit. \\
		I10 & Implement software hardening & l & Impl. \\
		D02 & Finish user documentation & m & Admin. \\
		D03 & Finish project documentation & h & Admin. \\ 
		D04 & Create project presentation & h & Admin. \\
		D05 & Create project poster & m & Admin. \\
		D06 & Create project video & l & Admin. \\
    \end{tabular}
  \end{center}
\end{table}

\section{Planning}

For the planning of this project the following milestones were created. Each coveres multiple workpackages. The mapping can be seen in table \ref{tab:milestones}. The milestones can also be seen in figure \ref{apdx-fig:milestones}. There they are displayed with an assumed and actual finish date.



\begin{table}[!ht]
  \begin{center}
    \caption{Milestones}
    \label{tab:milestones}
    \begin{tabular}{c|l|c|l}
      \textbf{ID} & \textbf{Short description} & \textbf{Workpackages} \\
      \hline
			00 & Setup & D00, D01, A00, A01, I00 \\
			01 & Initial functionality & A02, I01, I02, I03 \\
			02 & Rules & A03, I04, I05, V00 \\
			03 & Alerting & A04, I06, I07 \\
			04 & Exhaustive testing & V01, V02, V03 \\
			05 & Usability & I08, I09, D02 \\
			06 & Software Hardening & A05, I10 \\
			07 & Presentation & D02, D03, D04, D05, D06 \\
    \end{tabular}
  \end{center}
\end{table}


\begin{figure}[ht]
	\begin{ganttchart}[
		hgrid,
		vgrid,
		x unit=7mm,
		y unit chart=10mm,
		milestone label font = \footnotesize
		]{1}{17}
		\gantttitle{2019}{17}\\
		\gantttitlelist{1,...,17}{1}\\
		
		\ganttplannedmilestone{Setup}{4}
		\ganttactualmilestone{}{0}\\
		\ganttplannedmilestone{Initial functionality}{5}
		\ganttactualmilestone{}{0}\\
		\ganttplannedmilestone{Rules}{7}
		\ganttactualmilestone{}{0}\\
		\ganttplannedmilestone{Alerting}{8}
		\ganttactualmilestone{}{0}\\
		\ganttplannedmilestone{Exhaustive testing}{10}
		\ganttactualmilestone{}{0}\\
		\ganttplannedmilestone{Usability}{13}
		\ganttactualmilestone{}{0}\\
		\ganttplannedmilestone{Software Hardening}{15}
		\ganttactualmilestone{}{0}\\
		\ganttplannedmilestone{Presentation}{17}
		\ganttactualmilestone{}{17}
	\end{ganttchart}
	\caption{Milestones}
	\label{apdx-fig:milestones}
\end{figure}

\chapter{Implementation Decisions}
\label{sec:decisions}

For this thesis there were some things that had to be defined during the thesis. Some decisions were already made before the thesis began. Those were mainly to only focus on the \gls{fs} and to use \gls{tsk}. Most other things were deliberately left open, so that they could still change if need be. The biggest decisions, which are listed here, were the programming language, the language and format of the configuration, the \gls{dbms} and the output. Those are described in the following sections. 

Each of the decisions were made using a table with the criteria and the candidates. Each candidate has a ranking of 1-5 for each criteria where 1 is the lowest and 5 is the highest ranking. The rankings are then summed up and the highest candidate was chosen. 

\section{Programming Language}
\label{sec:decisions:language}

For the programming language any general purpose language would be applicable. I focus mostly on Python, Java, Go and C/C++. Those languages were chosen to further look into based on my knowledge, fit for the job and interest. There are some criteria on which I evaluated them listed below.

\begin{itemize}
    \item Fit for the system
    \item Investigation community acceptance / community resources
    \item Personal experience and interest
    \item Learning potential
\end{itemize}

Those categories were not chosen at random. Each plays an important part in a successful thesis. The fit for the system is a generally important part. Some languages like C\# don't really fit the system because it has historically been a windows only language. Although this is less true now, but because of that I excluded that specific language.

The acceptance of the investigation community has a direct link to the amount of community resources available. It is important that there exists a library that can be used to access the \gls{tsk} \gls{api}. If this was not the case I would need to create the bindings in the thesis which would be a lot of extra work. Additionally, if I use an exotic language, the chance that the system is actually used is smaller, because there will be no community backing it.

Personal experience and interest plays a role in any software development project. If the developer has no interest or experience in a language it takes longer until anything is completed. This is doubly true if the developer works alone. 

Learning potential is also important. The developer, me, is more engaged if he can learn something while writing the code and thus is faster. This might be counterintuitive, as he needs more time because there are some unknowns. However, challenging oneself and learning is a big part of any developer.

In table \ref{tab:dec:language} I evaluated the mentioned languages on those attributes. Python achieves the highest score with 21. This is mainly because on the one hand it has a lot of community support and an active library to interact with \gls{tsk}. Additionally, I already have some experience in it but not to much. This way starting off is easier and I can still learn a lot. 

While Go and C/C++ interest me and would be a rather good fit, I have nearly no experience in either of them. Additionally, go doesn't have any community resources and bindings for \gls{tsk}. For C/C++ what mostly brought me to give them such low rankings is the fact, that they are very easily exploitable. 

With Java I have a lot of experience, also it has a reasonably good fit and comunity. However, the motivation to do yet another java project was what lead me against this language. Additionally, the investigation community currently really likes python and not java.

\begin{table}[!ht]
    \begin{center}
        \caption{Comparison of programming languages}
        \label{tab:dec:language}
        \begin{tabular}{l|c|c|c|c|c|c}
            \textbf{Name} & \textbf{Fit} & \textbf{Community} & \textbf{Experience} & \textbf{Interest} & \textbf{Learning} & \textbf{Total}\\
        \hline
            Python  & 4 & 5 & 4 & 4 & 4 & 21 \\
            Go      & 5 & 1 & 1 & 4 & 5 & 16 \\
            C/C++   & 3 & 3 & 2 & 3 & 5 & 16 \\
            Java    & 4 & 3 & 5 & 1 & 1 & 14 \\
        \end{tabular}
    \end{center}
\end{table}



\section{Language for Configuration File}
\label{sec:decisions:config:language}

The language of the configuration file is at the same time less and more important than the programming language. For the developer it is less important. As long as it is managable it is ok. It needs good integration with the programming language and then everything would be fine. On the other hand, for the user it is more important. He does not care which language the software was written in. However, he might need to change the configuration file relatively often. 

I tried to bring both needs together. The categories that I used here are integration with python, readability, writeability and prevalence. The read and writeability is the amount of work that a user has to put in to read or write a configuration. This includes the likelyhood of syntax errors and the overall awkwardnes of the language. The prevalence is about how well know the configuration language is. This is important because it increases all three other parts.

For the languages that are evaluated I chose \gls{yaml}, \gls{json}, ini and \gls{xml}. Those are the recommended languages for python. In table \ref{tab:dec:config:language} I evaluated each language for each attribute. 

\gls{yaml} emerges victorious. This is mostly because it was made to be human readable. Other than \gls{xml} and \gls{json}. It is also easily extensible and does not require much syntax knowledge. \gls{xml} and \gls{json} contain a lot of additional symbols that need to be defined correctly. For configuration that is mainly for humans this is a big drawback. The INI format could also work. However, this format is not widely used.

\begin{table}[!ht]
    \begin{center}
        \caption{Comparison of configuration languages}
        \label{tab:dec:config:language}
        \begin{tabular}{l|c|c|c|c|c}
            \textbf{Name} & \textbf{Integration} & \textbf{readability} & \textbf{writeability} & \textbf{prevalence} & \textbf{Total}\\
        \hline
            YAML    & 4 & 5 & 5 & 5 & 19 \\
            JSON    & 4 & 5 & 3 & 5 & 17 \\
            XML     & 3 & 5 & 1 & 3 & 12 \\
            INI     & 5 & 3 & 3 & 1 & 12 \\
        \end{tabular}
    \end{center}
\end{table}

\section{Database Management System}
\label{sec:decisions:dbms}

The \gls{dbms} used has more requirements than the programming and configuration languages. There are many competeing \gls{dbms} which all would be viable. This decision is focussed on the primary \gls{dbms} used in this implementation, meaning the one that will be used in the thesis. The system should afterwards be extended to support many, if not most, of the different \gls{dbms} to give the user the possibility to configure the system the way that makes the most sense for them. 

The criteria for the first \gls{dbms} are simplicity, performance, support for prototyping, attack surface and usability on different systems. The \gls{dbms} analyzed all need to be \gls{opensource} or at least free of royalties. Chosen were \gls{sqlite}, MariaDB and PostgreSQL. As a note, only relational databases were chosen because of my experience. 

In table \ref{tab:dec:dbms} the results can be seen. \gls{sqlite} has won mainly because it has a high simplicity and good support for prototyping. Additionally, it also works on systems that need to be extremely encapsulated because of the data they store. This way it is harder to exfiltrate data from those high security systems. As said before, more \gls{dbms} should be added to the system after the thesis ends. Other \gls{dbms} have benefits especially in the performance, but \gls{sqlite} was chosen as dbms for the implementation in this thesis.

\begin{table}[!ht]
    \begin{center}
        \caption{Comparison of \gls{dbms}}
        \label{tab:dec:dbms}
        \begin{tabular}{l|c|c|c|c|c|c}
            \textbf{Name} & \textbf{Simplicity} & \textbf{Performance} & \textbf{Prototyping} & \textbf{Attack Surface} & \textbf{Usability} & \textbf{Total}\\
        \hline
            SQLite      & 5 & 2 & 5 & 5 & 5 & 22 \\
            PostgreSQL  & 3 & 5 & 3 & 3 & 2 & 16 \\
            MariaDB     & 3 & 4 & 3 & 3 & 2 & 15 \\
        \end{tabular}
    \end{center}
\end{table}

\section{Output / Alerting}
\label{sec:decisions:output}

The output or alerting is simmilar to the \gls{dbms}. There is not one correct choice. Multiple different ways need to be implemented if the system should become popular. This decision was mostly about the way of alerting that was implemented in this thesis. After the completion, most of the alerting methods should be supported, but it does not make sense to implement many of them during the thesis to not sway to far from the main topic.

The criteria for the output are extensibility, complexity for implementation, complexity for configuration and support for low configuration. Extensibility is mostly about how easily can it be extended into another form of alerting. Complexity is mostly about how much time is needed to implement it and what requirements does it require on the host system. For instance, for sending mails, an smtp server needs to be configured. The support for low configuration goes into how the type of alerting can be used if the user did not change the configuration a lot, or at all. This is true for when he just installed the system and wants to try it out. 

The candidates for output and alerting are console output, rest interface, email and writing to a log file. The results can be seen in table \ref{tab:dec:output}. Interestingly, output to the console has won easily. The biggest advantage from that is that this output can be handed over to any other application or script. This means that all the other candidates can be done by writing a script for them. This increases the extensibility enormly. Also, the complexity for both implementation and configuration is extremely low. A good console output is required for prototyping anyway, using it as alerting by letting the user handing it over to another tool seems to be the most sensible approach. Still, in the future additional alerting functions should be built in.

\begin{table}[!ht]
    \begin{center}
        \caption{Comparison of output and alerting functions}
        \label{tab:dec:output}
        \begin{tabular}{l|c|c|c|c|c}
            \textbf{Name} & \textbf{Extensible} & \textbf{Complex Impl} & \textbf{Complex Conf} & \textbf{Support no conf} & \textbf{Total}\\
        \hline
            Console Output  & 5 & 5 & 5 & 5 & 20 \\
            Log File        & 2 & 5 & 4 & 4 & 15 \\
            Rest Interface  & 4 & 3 & 1 & 1 & 9 \\
            Email           & 1 & 3 & 3 & 1 & 8 \\
        \end{tabular}
    \end{center}
\end{table}
\chapter{Journal}
\label{sec:journal}

In this section I list a journal of my work with aproximate hours I worked on the thesis with the total that I have worked until then. Those numbers are not extremely exact, as I did not use a time measurement tool. I am mainly listing challenges I faced and decisions I made. I will also list the meetings I had with \gls{bn} and \gls{ab} as well as a short meeting summary. I will also list the workpackages that were worked on and, if applicable, completed. Additionally I will also list milestones that were reached, postponed or cancelled.

There is one section per week where I worked on the thesis. Each section starts with a table that is running. The columns are \gls{cwp}, \gls{rm}, \gls{hw}, \gls{thw} and whether there was a meeting or not.


\section{Week 01 18.02.-24.02.}
\label{sec:journal:week01}

\begin{table}[!ht]
    \begin{center}
        \caption{Week 01}
        \label{tab:journal:week01}
        \begin{tabular}{l|c|c|c|c|c}
            \textbf{Week} & \textbf{\gls{cwp}} & \textbf{\gls{rm}} & \textbf{\gls{hw}} & \textbf{\gls{thw}} & \textbf{Meeting}\\
        \hline
        01 & - & - & 8 & 8 & Yes \\
        \end{tabular}
    \end{center}
\end{table}

\subsubsection{Tasks}

In the first week I mainly gathered information about the thesis itself. On the 18.02. was the official kickoff event from the \gls{bfh}. I also started gathering more information about \gls{hids}. I scheduled a meeting with \gls{bn} for the 01.03.

\subsubsection{Problems}

I did not face any problems in the first week.

\section{Week 02 25.02.-03.03.}
\label{sec:journal:week02}

\begin{table}[!ht]
    \begin{center}
        \caption{Week 02}
        \label{tab:journal:week02}
        \begin{tabular}{l|c|c|c|c|c}
            \textbf{Week} & \textbf{\gls{cwp}} & \textbf{\gls{rm}} & \textbf{\gls{hw}} & \textbf{\gls{thw}} & \textbf{Meeting}\\
        \hline
        01 & - & - & 8 & 8 & Yes \\
        02 & D00 & - & 12 & 20 & Yes \\
        \end{tabular}
    \end{center}
\end{table}

\subsubsection{Tasks}

This week the unofficial kickoff with \gls{bn} took place. We met in Bern and talked a lot about how we want to manage the project. The full meeting notes are available in section \ref{sec:meeting01}. I also took some time after the meeting for postprocessing of the meeting. Then I started by creating the initial draft of the \LaTeX documentation by taking the \gls{bfh} template and scaling it down to my needs. I also created a rough draft of the high level requirements and created a \href{https://github.com/Tartori/hids_thesis}{github repository}. 

\subsubsection{Problems}

The \LaTeX document was initially to complex and I did not understand all the packages. After some trial and error and reinstalling of packages I was able to get it to run. 

\section{Week 03 04.03.-10.03.}
\label{sec:journal:week03}
\begin{table}[!ht]
    \begin{center}
        \caption{Week 03}
        \label{tab:journal:week03}
        \begin{tabular}{l|c|c|c|c|c}
            \textbf{Week} & \textbf{\gls{cwp}} & \textbf{\gls{rm}} & \textbf{\gls{hw}} & \textbf{\gls{thw}} & \textbf{Meeting}\\
        \hline
        01 & - & - & 8 & 8 & Yes \\
        02 & D00 & - & 12 & 20 & Yes \\
        03 & - & - & 12 & 32 & Yes \\
        \end{tabular}
    \end{center}
\end{table}

\subsubsection{Tasks}

The main task that I did this week was the problem statement and the requirements that I needed to prepare for the second meeting. This is not in the workpackages. Then I needed to prepare for the meeting and postprocess it again. The meeting notes are in section \ref{sec:meeting02}. Overall, I continued gathering information about \gls{hids}, \gls{fs} and \gls{nids}.

\subsubsection{Problems}

I had no problems in week three.

\section{Week 04 11.03.-17.03.}
\label{sec:journal:week04}

\begin{table}[!ht]
    \begin{center}
        \caption{Week 04}
        \label{tab:journal:week04}
        \begin{tabular}{l|c|c|c|c|c}
            \textbf{Week} & \textbf{\gls{cwp}} & \textbf{\gls{rm}} & \textbf{\gls{hw}} & \textbf{\gls{thw}} & \textbf{Meeting}\\
        \hline
        01 & - & - & 8 & 8 & Yes \\
        02 & D00 & - & 12 & 20 & Yes \\
        03 & - & - & 12 & 32 & Yes \\
        04 & D01, A00 & - & 24 & 56 & No \\
        \end{tabular}
    \end{center}
\end{table}

\subsubsection{Tasks}

This week I wanted to improve the \LaTeX documentation. Additionally, I started working on the workpackages and milestones.

For non project management tasks, I started researching other \gls{hids} that are available. Mainly I focussed on tripwire and aide. I also started researching \gls{tsk} and its \gls{api}

\subsubsection{Problems}

\begin{itemize}
    \item Splitting the work into actionable workpackages
    \item Table formating in \LaTeX
    \item Milestone formatting in \LaTeX
    \item Almost no publicly available information on tripwire.

\end{itemize}

\section{Week 05 18.03.-24.03.}
\label{sec:journal:week05}

\begin{table}[!ht]
    \begin{center}
        \caption{Week 05}
        \label{tab:journal:week05}
        \begin{tabular}{l|c|c|c|c|c}
            \textbf{Week} & \textbf{\gls{cwp}} & \textbf{\gls{rm}} & \textbf{\gls{hw}} & \textbf{\gls{thw}} & \textbf{Meeting}\\
        \hline
        01 & - & - & 8 & 8 & Yes \\
        02 & D00 & - & 12 & 20 & Yes \\
        03 & - & - & 12 & 32 & Yes \\
        04 & D01, A00 & - & 24 & 56 & No \\
        05 & A01 & - & 24 & 80 & Yes \\
        \end{tabular}
    \end{center}
\end{table}

\subsubsection{Tasks}

With the knowledge of \gls{tsk} I was able to search for libraries that support the \gls{tsk} \gls{api} for the different languages that I wanted to evaluate and then I decided the language. Before the meeting, I decided on the programming language, Python. The full reasoning is available in section \ref{sec:decisions:language}.

I also had to prepare for the meeting this week. The outline is shown in section \ref{sec:meeting03}.

\subsubsection{Problems}

\begin{itemize}
    \item Finding good \gls{tsk} libraries was not possible for all languages
    \item Tutorial on Go was harder than anticipated.
\end{itemize}

\section{Week 06 25.03.-31.03.}
\label{sec:journal:week06}

\begin{table}[!ht]
    \begin{center}
        \caption{Week 06}
        \label{tab:journal:week06}
        \begin{tabular}{l|c|c|c|c|c}
            \textbf{Week} & \textbf{\gls{cwp}} & \textbf{\gls{rm}} & \textbf{\gls{hw}} & \textbf{\gls{thw}} & \textbf{Meeting}\\
        \hline
        01 & - & - & 8 & 8 & Yes \\
        02 & D00 & - & 12 & 20 & Yes \\
        03 & - & - & 12 & 32 & Yes \\
        04 & D01, A00 & - & 24 & 56 & No \\
        05 & A01 & - & 24 & 80 & Yes \\
        06 & I00 & 00 & 12 & 92 & Yes \\
        \end{tabular}
    \end{center}
\end{table}

\subsubsection{Tasks}

This week I started by contacting \gls{ab}. We then decided for a meeting in the same week. I then used time to prepare for said meeting. Additionally, I started to create a local developer environment with the correct libraries and tools. With that I completed Milestone 00. Then I had the meeting with \gls{ab} which needed a lot of postprocessing. The meeting notes are available in section \ref{sec:meetingexpert}.

\subsubsection{Problems}

\begin{itemize}
    \item I needed a lot of time to research certain uncertanties that appeared after talking with \gls{ab}
\end{itemize}


\section{Week 07 01.04.-07.04.}
\label{sec:journal:week07}

\begin{table}[!ht]
    \begin{center}
        \caption{Week 07}
        \label{tab:journal:week07}
        \begin{tabular}{l|c|c|c|c|c}
            \textbf{Week} & \textbf{\gls{cwp}} & \textbf{\gls{rm}} & \textbf{\gls{hw}} & \textbf{\gls{thw}} & \textbf{Meeting}\\
        \hline
        01 & - & - & 8 & 8 & Yes \\
        02 & D00 & - & 12 & 20 & Yes \\
        03 & - & - & 12 & 32 & Yes \\
        04 & D01, A00 & - & 24 & 56 & No \\
        05 & A01 & - & 24 & 80 & Yes \\
        06 & I00 & 00 & 12 & 92 & Yes \\
        07 & - & - & 32 & 124 & No \\
        \end{tabular}
    \end{center}
\end{table}

\subsubsection{Tasks}

This week I realized that I am already behind schedule. I started by going through the goals and rework some of them to increase the clarity. I also rushed into trying to get the initial functionality going. This was not as easy as I anticipated and I lost a lot of time trying to get the output of \gls{pytsk} that I needed. This was tough and I frequently checked the source code of \gls{pytsk} to find solutions. 

\subsubsection{Problems}

\begin{itemize}
    \item I was not able to find the correct way to use \gls{pytsk} to walk through the \gls{fs}
\end{itemize}

\section{Week 08 08.04.-14.04.}
\label{sec:journal:week08}

\begin{table}[!ht]
    \begin{center}
        \caption{Week 08}
        \label{tab:journal:week08}
        \begin{tabular}{l|c|c|c|c|c}
            \textbf{Week} & \textbf{\gls{cwp}} & \textbf{\gls{rm}} & \textbf{\gls{hw}} & \textbf{\gls{thw}} & \textbf{Meeting}\\
        \hline
        01 & - & - & 8 & 8 & Yes \\
        02 & D00 & - & 12 & 20 & Yes \\
        03 & - & - & 12 & 32 & Yes \\
        04 & D01, A00 & - & 24 & 56 & No \\
        05 & A01 & - & 24 & 80 & Yes \\
        06 & I00 & 00 & 12 & 92 & Yes \\
        07 & - & - & 32 & 124 & No \\
        08 & - & - & 20 & 144 & Yes \\
        \end{tabular}
    \end{center}
\end{table}

\subsubsection{Tasks}

This week I started frustrated because I was not able to implement the file walking ability. However, after getting to it with a fresh mind I was able to find the solution by finding a new source of documentation. I was glad that I could find it. However, I still could not get all the information that I needed out of the \gls{pytsk} or rather I did not yet know how to get to all the data.

Then I started preparing for the meeting with both \gls{ab} and \gls{bn}.

\subsubsection{Problems}

\begin{itemize}
    \item I could not get to all the data.
\end{itemize}

\section{Week 09 15.04.-21.04.}
\label{sec:journal:week09}

\begin{table}[!ht]
    \begin{center}
        \caption{Week 09}
        \label{tab:journal:week09}
        \begin{tabular}{l|c|c|c|c|c}
            \textbf{Week} & \textbf{\gls{cwp}} & \textbf{\gls{rm}} & \textbf{\gls{hw}} & \textbf{\gls{thw}} & \textbf{Meeting}\\
        \hline
        01 & - & - & 8 & 8 & Yes \\
        02 & D00 & - & 12 & 20 & Yes \\
        03 & - & - & 12 & 32 & Yes \\
        04 & D01, A00 & - & 24 & 56 & No \\
        05 & A01 & - & 24 & 80 & Yes \\
        06 & I00 & 00 & 12 & 92 & Yes \\
        07 & - & - & 32 & 124 & No \\
        08 & - & - & 20 & 144 & Yes \\
        09 & - & - & 4 & 148 & No \\
        \end{tabular}
    \end{center}
\end{table}

\subsubsection{Tasks}

This week was a very low week. I was frustrated by the inability to get \gls{pytsk} to work the way I wanted and didn't have much motivation to work on the documentation. I took it slow and would regret it in the following weeks.

\subsubsection{Problems}

\begin{itemize}
    \item I had issues with motivation.
\end{itemize}

\section{Week 10 22.04.-28.04.}
\label{sec:journal:week10}

\begin{table}[!ht]
    \begin{center}
        \caption{Week 10}
        \label{tab:journal:week10}
        \begin{tabular}{l|c|c|c|c|c}
            \textbf{Week} & \textbf{\gls{cwp}} & \textbf{\gls{rm}} & \textbf{\gls{hw}} & \textbf{\gls{thw}} & \textbf{Meeting}\\
        \hline
        01 & - & - & 8 & 8 & Yes \\
        02 & D00 & - & 12 & 20 & Yes \\
        03 & - & - & 12 & 32 & Yes \\
        04 & D01, A00 & - & 24 & 56 & No \\
        05 & A01 & - & 24 & 80 & Yes \\
        06 & I00 & 00 & 12 & 92 & Yes \\
        07 & - & - & 32 & 124 & No \\
        08 & - & - & 20 & 144 & Yes \\
        09 & - & - & 4 & 148 & No \\
        10 & I01 & - & 20 & 168 & No \\
        \end{tabular}
    \end{center}
\end{table}

\subsubsection{Tasks}

This week I knew that I had to finish the file scanning ability. Especially after slacking off the week before I started researching and trying until I found the correct search query which lead me to the documentation of \gls{tsk} \gls{api} for the file struct. I felt very stupid because I had not found it before. With this documentation I could make sense of the output of \gls{pytsk} and I was able to create an appropriate python class with the most important fields of a \gls{tsk} file.  


\subsubsection{Problems}

\begin{itemize}
    \item At first I could not find the correct documentation.
    \item I struggeled with the datatypes of how the permissions were stored before I realized that it is a base 10 representation of the base8 permissions.
    \item Debugging was quite a challenge since the code needs root access to connect to the local harddrive.
\end{itemize}

\section{Week 11 29.04.-05.05.}
\label{sec:journal:week11}

\begin{table}[!ht]
    \begin{center}
        \caption{Week 11}
        \label{tab:journal:week11}
        \begin{tabular}{l|c|c|c|c|c}
            \textbf{Week} & \textbf{\gls{cwp}} & \textbf{\gls{rm}} & \textbf{\gls{hw}} & \textbf{\gls{thw}} & \textbf{Meeting}\\
        \hline
        01 & - & - & 8 & 8 & Yes \\
        02 & D00 & - & 12 & 20 & Yes \\
        03 & - & - & 12 & 32 & Yes \\
        04 & D01, A00 & - & 24 & 56 & No \\
        05 & A01 & - & 24 & 80 & Yes \\
        06 & I00 & 00 & 12 & 92 & Yes \\
        07 & - & - & 32 & 124 & No \\
        08 & - & - & 20 & 144 & Yes \\
        09 & - & - & 4 & 148 & No \\
        10 & I01 & - & 20 & 168 & No \\
        11 & A02, I02, I03 & 01 & 32 & 200 & Yes \\
        \end{tabular}
    \end{center}
\end{table}

\subsubsection{Tasks}

After feeling the relief from last week and the pressure from being rather far behind I really wanted to finish Milestone 01. I did this by creating the database, a configuration to connect to the database and then storing the results of the file walk. For both the configuration and the database I had to first define which I wanted to use. For the configuration it was clear pretty fast that \gls{yaml} seems the appropriate choice. For the \gls{dbms} I was more unsure. I first wanted to implement both, \gls{sqlite} and postgres. However, I could not bring postgres to run on my developer environment. Considering the time and the other functionality, I decided to drop postgres and focus on \gls{sqlite}.

Besides that, I also had another meeting. I really wanted to show my first prototype and thus I forced throguh all hardships. I also had other things to prepare for the meeting, but finally I was able to gain valuable feedback for my prototype. The notes for this meeting are listed in section \ref{sec:meeting05}

Also, I was able to fix the debugging challenge. I simply created an image of an usb stick on which I then run the code. This had two advantages. Firstly, it did not need root access. Secondly, it was faster since the stick was smaller than the harddisk.

\subsubsection{Problems}

\begin{itemize}
    \item Postgres would not run on local machine. 
    \item Database schema needed some trial and error until it worked properly.
\end{itemize}

\section{Week 12 06.05.-12.05.}
\label{sec:journal:week12}

\begin{table}[!ht]
    \begin{center}
        \caption{Week 12}
        \label{tab:journal:week12}
        \begin{tabular}{l|c|c|c|c|c}
            \textbf{Week} & \textbf{\gls{cwp}} & \textbf{\gls{rm}} & \textbf{\gls{hw}} & \textbf{\gls{thw}} & \textbf{Meeting}\\
        \hline
        01 & - & - & 8 & 8 & Yes \\
        02 & D00 & - & 12 & 20 & Yes \\
        03 & - & - & 12 & 32 & Yes \\
        04 & D01, A00 & - & 24 & 56 & No \\
        05 & A01 & - & 24 & 80 & Yes \\
        06 & I00 & 00 & 12 & 92 & Yes \\
        07 & - & - & 32 & 124 & No \\
        08 & - & - & 20 & 144 & Yes \\
        09 & - & - & 4 & 148 & No \\
        10 & I01 & - & 20 & 168 & No \\
        11 & A02, I02, I03 & 01 & 32 & 200 & Yes \\
        12 & - & - & 20 & 220 & No \\
        \end{tabular}
    \end{center}
\end{table}

\subsubsection{Tasks}

In this week I wanted to implement the changes proposed by \gls{bn}. I did this by first adding all the information to the python class and then adding it to the database. This actually required some trial and error since it was not always easy to find the correct values. Also, I did it twice, since I forgot to push it to the github repo and then was not sure if I already did it on the second machine. 

\subsubsection{Problems}

\begin{itemize}
    \item I did the work twice because I forgot to push it on my home pc.
    \item The USB Stick Image workaround did not work for everything as the \gls{fs} on it did not contain all required attributes.
    \item Sometimes I had issues with the Python module handling. I was not always able to import the classes correctly.
\end{itemize}

\section{Week 13 13.05.-19.05.}
\label{sec:journal:week13}

\begin{table}[!ht]
    \begin{center}
        \caption{Week 13}
        \label{tab:journal:week13}
        \begin{tabular}{l|c|c|c|c|c}
            \textbf{Week} & \textbf{\gls{cwp}} & \textbf{\gls{rm}} & \textbf{\gls{hw}} & \textbf{\gls{thw}} & \textbf{Meeting}\\
        \hline
        01 & - & - & 8 & 8 & Yes \\
        02 & D00 & - & 12 & 20 & Yes \\
        03 & - & - & 12 & 32 & Yes \\
        04 & D01, A00 & - & 24 & 56 & No \\
        05 & A01 & - & 24 & 80 & Yes \\
        06 & I00 & 00 & 12 & 92 & Yes \\
        07 & - & - & 32 & 124 & No \\
        08 & - & - & 20 & 144 & Yes \\
        09 & - & - & 4 & 148 & No \\
        10 & I01 & - & 20 & 168 & No \\
        11 & A02, I02, I03 & 01 & 32 & 200 & Yes \\
        12 & - & - & 20 & 220 & No \\
        13 & A03, I04, I05, V00 & 02 & 32 & 252 & No \\
        \end{tabular}
    \end{center}
\end{table}

\subsubsection{Tasks}

In this week I was able to get a lot of momentum. Firstly I created a new directory for the system. This way the thesis documentation and the \gls{hids} were split. Then I looked deeper into python modules and how to manage the imports properly. By doing that I was able to fix some issues of files sometimes not getting imported correctly. Also, I found some more information in the \gls{tsk} \gls{api} which I had not seen before. I added this information. Also, I added the errors that were thrown as to not lose information. I then followed with a bigger testing period on my local machine. This resulted in frustratingly many errors, which as I later found out were a direct result of how my filewalking process worked. I was able to solve all those errors and then added the first draft of finding intrusions. This config was harder to draft, but I came up with a implementation that would cover many usecases. I also implemented a first draft of the investigator, which for the moment takes a very long time.

In project management view I created a first draft of the thesis documentation and prepared for the meeting that would be on the start of next week.

\subsubsection{Problems}

\begin{itemize}
    \item The system took about 15 minutes to complete on my local machine which resulted in long waiting periods
    \item The investigator config was rather hard to design
    \item The investigator took extremely long itself
\end{itemize}

\section{Week 14 20.05.-26.05.}
\label{sec:journal:week14}

\begin{table}[!ht]
    \begin{center}
        \caption{Week 14}
        \label{tab:journal:week14}
        \begin{tabular}{l|c|c|c|c|c}
            \textbf{Week} & \textbf{\gls{cwp}} & \textbf{\gls{rm}} & \textbf{\gls{hw}} & \textbf{\gls{thw}} & \textbf{Meeting}\\
        \hline
        01 & - & - & 8 & 8 & Yes \\
        02 & D00 & - & 12 & 20 & Yes \\
        03 & - & - & 12 & 32 & Yes \\
        04 & D01, A00 & - & 24 & 56 & No \\
        05 & A01 & - & 24 & 80 & Yes \\
        06 & I00 & 00 & 12 & 92 & Yes \\
        07 & - & - & 32 & 124 & No \\
        08 & - & - & 20 & 144 & Yes \\
        09 & - & - & 4 & 148 & No \\
        10 & I01 & - & 20 & 168 & No \\
        11 & A02, I02, I03 & 01 & 32 & 200 & Yes \\
        12 & - & - & 20 & 220 & No \\
        13 & A03, I04, I05, V00 & 02 & 28 & 252 & No \\
        14 & - & - & 16 & 268 & Yes \\
        \end{tabular}
    \end{center}
\end{table}

\subsubsection{Tasks}

In this week I started writing in the documentation by creating the first draft of the introduction. 

However, I mainly focussed in bringing the scanning time down. I tried many approaches and wanted to find out where the issue lies. After clarifying with the main developer of \gls{pytsk} it was clear that a direct filewalk on the disk image was not possible. However, on the same day I found a slightly tweaked version of my implementation that would result in an incredible performance boost. I no longer would use the directory paths but the directories that I already had. This way the runtime of the scanner went from about 15 minutes on my home machine to about 30 seconds. Sadly, I only realized that after a longer time. At first I had the investigator running as well as the scanner. This resulted in extremely long runtime. After I deactivated the investigator I realized how much faster it had become. 

\subsubsection{Problems}

\begin{itemize}
    \item By running both the scanner and the investigator I did not initially see the performance boost of the scanner.
    \item \gls{pytsk} does not support a direct file walk.
\end{itemize}

\section{Week 15 27.05.-02.06.}
\label{sec:journal:week15}

\begin{table}[!ht]
    \begin{center}
        \caption{Week 15}
        \label{tab:journal:week15}
        \begin{tabular}{l|c|c|c|c|c}
            \textbf{Week} & \textbf{\gls{cwp}} & \textbf{\gls{rm}} & \textbf{\gls{hw}} & \textbf{\gls{thw}} & \textbf{Meeting}\\
        \hline
        01 & - & - & 8 & 8 & Yes \\
        02 & D00 & - & 12 & 20 & Yes \\
        03 & - & - & 12 & 32 & Yes \\
        04 & D01, A00 & - & 24 & 56 & No \\
        05 & A01 & - & 24 & 80 & Yes \\
        06 & I00 & 00 & 12 & 92 & Yes \\
        07 & - & - & 32 & 124 & No \\
        08 & - & - & 20 & 144 & Yes \\
        09 & - & - & 4 & 148 & No \\
        10 & I01 & - & 20 & 168 & No \\
        11 & A02, I02, I03 & 01 & 32 & 200 & Yes \\
        12 & - & - & 20 & 220 & No \\
        13 & A03, I04, I05, V00 & 02 & 28 & 252 & No \\
        14 & - & - & 16 & 268 & Yes \\
        15 & D05 & - & 40 & 308 & No \\
        \end{tabular}
    \end{center}
\end{table}

\subsubsection{Tasks}

This week I started focussing on the documentation. So far this was not a priority for me. I took all week to come up with a version of the documentation that would be viable as a first draft. I also created the Poster in this week. 

\subsubsection{Problems}

Besides some \LaTeX issues everything went fine this week. (Appart from the obvious issue that I should have started with this much earlier...)

\section{Week 16 03.06.-09.06.}
\label{sec:journal:week16}


\begin{table}[!ht]
    \begin{center}
        \caption{Week 16}
        \label{tab:journal:week16}
        \begin{tabular}{l|c|c|c|c|c}
            \textbf{Week} & \textbf{\gls{cwp}} & \textbf{\gls{rm}} & \textbf{\gls{hw}} & \textbf{\gls{thw}} & \textbf{Meeting}\\
        \hline
        01 & - & - & 8 & 8 & Yes \\
        02 & D00 & - & 12 & 20 & Yes \\
        03 & - & - & 12 & 32 & Yes \\
        04 & D01, A00 & - & 24 & 56 & No \\
        05 & A01 & - & 24 & 80 & Yes \\
        06 & I00 & 00 & 12 & 92 & Yes \\
        07 & - & - & 32 & 124 & No \\
        08 & - & - & 20 & 144 & Yes \\
        09 & - & - & 4 & 148 & No \\
        10 & I01 & - & 20 & 168 & No \\
        11 & A02, I02, I03 & 01 & 32 & 200 & Yes \\
        12 & - & - & 20 & 220 & No \\
        13 & A03, I04, I05, V00 & 02 & 28 & 252 & No \\
        14 & - & - & 16 & 268 & Yes \\
        15 & D05 & - & 40 & 308 & No \\
        16 & A03, I04, I05, V00, I06, A04, I07, V01 & 03, (04) & 40 & 348 & Yes \\
        \end{tabular}
    \end{center}
\end{table}

\subsubsection{Tasks}

This week I focussed on both. First I wanted to bring the \gls{hids} to an appropriate end. I did this by first redefining the rules and investigations to better match what is required. I adjusted them several times until they resulted in the final version that is available now. Additionally, I reworked the way the investigator finds relations between the files. By passing that task to the database I could gain increadible performance increases. I also verified the system by letting it run and then changing files. This way I reworked milestone 02, implemented milestone 03 and partially implemented milestone 04. 

In the documentation I also changed a lot. I added many chapters while improving most already created ones. I also created the abstract and added that to the book tool from the \gls{bfh}.

\subsubsection{Problems}

This week was very productive and I did not really have any big issues. After so much time working with \gls{pytsk} and \gls{tsk} I was able to quickly maneuver through the code. Also changing the milestone 02 was not that big after I realized what the biggest issues were. 

In short, I did not really have issues.

\section{Week 17 10.06.-16.06.}
\label{sec:journal:week17}

\begin{table}[!ht]
    \begin{center}
        \caption{Week 17}
        \label{tab:journal:week17}
        \begin{tabular}{l|c|c|c|c|c}
            \textbf{Week} & \textbf{\gls{cwp}} & \textbf{\gls{rm}} & \textbf{\gls{hw}} & \textbf{\gls{thw}} & \textbf{Meeting}\\
        \hline
        01 & - & - & 8 & 8 & Yes \\
        02 & D00 & - & 12 & 20 & Yes \\
        03 & - & - & 12 & 32 & Yes \\
        04 & D01, A00 & - & 24 & 56 & No \\
        05 & A01 & - & 24 & 80 & Yes \\
        06 & I00 & 00 & 12 & 92 & Yes \\
        07 & - & - & 32 & 124 & No \\
        08 & - & - & 20 & 144 & Yes \\
        09 & - & - & 4 & 148 & No \\
        10 & I01 & - & 20 & 168 & No \\
        11 & A02, I02, I03 & 01 & 32 & 200 & Yes \\
        12 & - & - & 20 & 220 & No \\
        13 & A03, I04, I05, V00 & 02 & 28 & 252 & No \\
        14 & - & - & 16 & 268 & Yes \\
        15 & D05 & - & 40 & 308 & No \\
        16 & A03, I04, I05, V00, I06, A04, I07, V01 & 03, (04) & 40 & 348 & Yes \\
        17 & A05, I10, D03, D04 & 06, 07 & 40 & 388 & Yes \\
        \end{tabular}
    \end{center}
\end{table}

\subsubsection{Tasks}

This week I needed to finish both, thesis and the system.

For the system, I implemented the python packaging guide. This way the system is available through the python packaging system. I also added a \gls{pgp} signature with a new private public keypair. This way the software is at least somewhat hardened. 

For the documentation. I finished it. Also I finished the presentation and held it on the day after handing this documentation in. Hopefully it went well.

\subsubsection{Problems}


\section{Week 18 17.06.-23.06.}
\label{sec:journal:week18}

\begin{table}[!ht]
    \begin{center}
        \caption{Week 18}
        \label{tab:journal:week18}
        \begin{tabular}{l|c|c|c|c|c}
            \textbf{Week} & \textbf{\gls{cwp}} & \textbf{\gls{rm}} & \textbf{\gls{hw}} & \textbf{\gls{thw}} & \textbf{Meeting}\\
        \hline
        01 & - & - & 8 & 8 & Yes \\
        02 & D00 & - & 12 & 20 & Yes \\
        03 & - & - & 12 & 32 & Yes \\
        04 & D01, A00 & - & 24 & 56 & No \\
        05 & A01 & - & 24 & 80 & Yes \\
        06 & I00 & 00 & 12 & 92 & Yes \\
        07 & - & - & 32 & 124 & No \\
        08 & - & - & 20 & 144 & Yes \\
        09 & - & - & 4 & 148 & No \\
        10 & I01 & - & 20 & 168 & No \\
        11 & A02, I02, I03 & 01 & 32 & 200 & Yes \\
        12 & - & - & 20 & 220 & No \\
        13 & A03, I04, I05, V00 & 02 & 28 & 252 & No \\
        14 & - & - & 16 & 268 & Yes \\
        15 & D05 & - & 40 & 308 & No \\
        16 & A03, I04, I05, V00, I06, A04, I07, V01 & 03, (04) & 40 & 348 & Yes \\
        17 & A05, I10 & 06, 07 & 40 & 388 & Yes \\
        18 & - & - & 16 & 404 & Yes \\
        \end{tabular}
    \end{center}
\end{table}

\subsubsection{Tasks}

It is actually hard to describe what I do this week, because as of writing this it is still in the future. What I need to do is prepare for the bachelor thesis defense and actually holding that. I assume to take 16 hours for preparation and for the defense itself. 

Besides that I also need to create a video of this thesis. 

\subsubsection{Problems}

\begin{itemize}
    \item I have no experience in video editing and creation. Thus, I expect this task to be quite challenging.
\end{itemize}
\chapter{User Documentation}
\label{sec:userdocu}

\chapter{Content of USB-Stick}
\label{chap:appendix_CDROM}

The enclosed USB-Stick contains two ZIP Files. One for the documentation and another for the source code. Both are also available in github. \href{https://github.com/Tartori/hids\_thesis}{Documentation: https://github.com/Tartori/hids\_thesis}, \href{https://github.com/Tartori/fids}{source: https://github.com/Tartori/fids}.

\begin{itemize}
  \item Documentation
  \begin{description}
    \item [img] Images used in the thesis and presentation
    \item [thesis] \LaTeX and pdf version of the thesis
    \item [presentation] \LaTeX and pdf version of the presentation for the techday
  \end{description}
  \item Source Code
  \begin{description}
    \item [fids] FIDS package with python sourcecode
  \end{description}
\end{itemize}


\newpage

%% Print the bibibliography and add the section to the table of content

\end{document}
