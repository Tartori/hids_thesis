\usepackage{glossaries}

\newacronym{cli}{CLI}{comand line interface}

\newacronym{ids}{IDS}{intrusion detection system}
\newacronym{hids}{HIDS}{host based intrusion detection system} 
\newacronym{nids}{NIDS}{network based intrusion detection system} 
\newacronym{nist}{NIST}{National Institute of Standards and Technology}
\newacronym{it}{IT}{Information Security}
\newacronym{tb}{TB}{Terrabyte}
\newacronym{mb}{MB}{Megabyte}
\newacronym{kb}{KB}{Kilobyte}
\newacronym{itsec}{ITSec}{\gls{it} Security}
\newglossaryentry{yaml}{name=YAML, description={YAML Ain't Markup Language}}
\newacronym{tsk}{TSK}{The Sleuth Kit}
\newacronym{tls}{TLS}{Transport Layer Security}
\newacronym{api}{API}{Application Programming Interface}
\newacronym{json}{JSON}{JavaScript Object Notation}
\newacronym{loa}{LoA}{Level of Assurance}
\newacronym{hdd}{HDD}{Hard Disk Drive}
\newacronym{ssd}{SSD}{Solid State Drive}
\newacronym{cd}{CD}{Compact Disc}
\newacronym{dvd}{DVD}{Digital Versatile Disc}
\newacronym{usb}{USB}{Universal Serial Bus}
\newacronym{ioc}{IoC}{Indicator of Compromise}
\newglossaryentry{sql}{name=SQL, description={A domain specific language used for querying a relational database.}}


\newglossaryentry{}{name=, description={}}
\newglossaryentry{postgres}{name=postgres, description={An opensource database server. It is rather lightweight and heavily used.}}
\newglossaryentry{sqlite}{name=sqlite, description={A very lightweight relational database implemnentation. It does not have a dedicated server but instead writes the database to a file which is written to the local host.}}
\newglossaryentry{opensource}{name=opensource, description={Software where both the source and the software is freely accessible and changable.}}
\newglossaryentry{github}{name=github, description={A platform for \gls{opensource} projects. It is free to use and hosts the source code for many projects.}}
\newglossaryentry{storagemedia}{name=storage media, description={media that is used to store data in a computer system.}}
\newglossaryentry{malware}{name=malware, description={Malware is any program that is designed to harm a computer system. The term includes well known terms like Virus, Worm, Adware, Keylogger, Trojan, etc. Malware usually tries to hide it's traces to achieve longer infection periods of time. }}
\newglossaryentry{anomaly}{
name=anomaly, 
plural=anomalies,
description={An unexpected change in the file system. For a change to be unexpected it needs to be covered by the configuration changes are unexpected only if the configuration says so. Some examples that might cause anomalies, changed rights, new files, deleted files. }}
\newglossaryentry{intrusion}{
name=intrusion, 
plural=intrusions,
description={An unauthorized access to a system or to data. }}
\newglossaryentry{investigation}{
name=investigation, 
plural=investigations,
description={The process of finding out what exactly happened after an incident.}}
\newglossaryentry{hash}{
name=cryptographic hash function, 
plural=cryptographic hash functions,
description={A deterministic one way function that fullfills collision resistance and other cryptographic properties. Implementations are the SHA-2 and SHA-3 families.}}
\newglossaryentry{nonhash}{
name=non-cryptographic hash function, 
plural=non-cryptographic hash functions,
description={A deterministic one way function that does not the hard to achieve properties that make a \gls{hash}. Often used when speed is more important than collision resistance.}}
\newglossaryentry{fs}{
name=file system, 
description={A file system is used to create a layer of abstraction between the hardware of the storage medium and the operating system. There are multiple file system types which are in use with different capabilities. Additionally to the files the file systems keep track of meta data to each file. This meta data includes some timestamps (created, last accessed, etc) and more.}}
\newglossaryentry{pytsk}{name=pytsk3, description={Python bindings for \gls{tsk}}}
\newglossaryentry{metadata}{name=metadata, description={Data that gives information on other data. In this thesis it is mostly used to describe attributes of the \gls{fs} that describes files. Exmaples are creation date and permissions.}}
\newglossaryentry{git}{name=git, description={A distributed version-control system used for source code tracking in software engineering. Nicknamed \'the stupid content tracker\', stands for either \'global information tracker\' or \'goddamn idiotic truckload of s...\'}}




