\usepackage{glossaries}


\newglossaryentry{anomaly}{
  name=anomaly, 
  description={An unexpected change in the file system. For a change to be unexpected it needs to be covered by the configuration changes are unexpected only if the configuration says so. Some examples that might cause anomalies, changed rights, new files, deleted files. }}
  \newglossaryentry{hash}{
    name=hash, 
    description={A mathematical function which defines one specific output per input that can not be reversed.}}
  \newacronym{hids}{HIDS}{host based intrusion detection system} 
\newacronym{nist}{NIST}{National Institute of Standards and Technology}
\newacronym{it}{IT}{Information Security}
\newacronym{itsec}{ITSec}{\gls{it} Security}
\newacronym{tls}{TLS}{Transport Layer Security}
\newacronym{api}{API}{Application Programming Interface}
\newacronym{json}{JSON}{JavaScript Object Notation}
\newglossaryentry{sha-256}{name=SHA-256, description={Secure Hash Algorithm 2 with 256 Bit Output}}
\newacronym{utf-8}{UTF-8}{Universal Transformation Format}
\newacronym{nfc}{NFC}{Near Field Communication}
\newacronym{usb}{USB}{Universal Serial Bus}
\newacronym{mitm}{MitM}{Man in the Middle}
\newacronym{loa}{LoA}{Level of Assurance}
\newglossaryentry{base64}{
name={Base 64},
  description={Base 64 is a binary-to-text encoding that uses 64 ASCII characters to represent 6 bit. This results in an overhead of 2 bit per byte. There is also a Base 64 URL encoding which uses slightly different characters.}}
\newglossaryentry{csa}{name={credential stuffing attacks}, description={Using credentials that were leaked on one site to access other sites. This is often an attack against passwords when passwords are reused}}

