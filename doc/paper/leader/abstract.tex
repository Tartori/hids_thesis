\chapter*{Abstract}
\label{chap:abstract}

Many tools exist to help a company to protect itself against cyber attacks. One type of such a tool is called an intrusion detection system (IDS). Those are created to detect attacks that somehow got through other measures and infected one or many hosts in the network. One type of IDS is called network-based intrusion detection system (NIDS). They operate on a network level and analyze the incoming and outgoing traffic for anomalies. Those anomalies usually signal an intrusion. When they find such an intrusion, they usually generate an alert for a system administrator or security professional to analyze. 

There is also another type of IDS called host-based intrusion detection system. They operate directly on the host and try to find attacks there. They are more effective at finding intrusions that are dormant and don't do anything for some time. They mostly operate on the file system and sometimes go beyond that. HIDS have been quite effective at finding intrusions on file basis in the past by creating cryptographic hashes of hashes and comparing them to previous executions. However, as file sizes have been growing, they began to struggle to execute within a short time frame. Calculating a hash is seen as the only reliable way to find changes to the file system, and with more data, it is taking increasingly long to calculate them. The situation has grown out of proportions because the time to scan now takes so long, that the intrusion detection system can't reliably find intrusions within a useful timespan.

In my thesis, I try to show a different solution to this problem. I created a host-based intrusion detection system that works at a file basis but does not calculate any hash. Instead, it finds intrusions by evaluating the file system attributes like modification time and permissions. This approach is risk-based because it is less reliable, but by increasing the speed, the host can be scanned multiple times more often than if hashes get calculated. I am using an open source forensic investigation tool called the sleuth kit (TSK). It offers much functionality for file system analysis and works on most operating systems. With this tool, I can extract the file system attributes reliably and fast, without touching the files themselves.

There is another advantage that I hope to give with my system. Forensic investigators usually struggle to reliably create a timeline of what happened on a file system after an intrusion. This timeline is essential because it can lead them to find out what exactly happened and often can make future intrusions harder. Here I want to help as well. Other than the other HIDS, my system stores all the executions. This way, an investigator can look at this history and sees how the attack started. One nice side-effect of that is that my system is very flexible. After changing something on the host, the system automatically adjusts.

With my tool, system administrators and forensic investigators have another option to tackle intrusion detection. Taking the risk-based approach can lead to many fast detections that otherwise would not be detected in time. Additionally, investigators have more data at their disposal to protect from future attacks.