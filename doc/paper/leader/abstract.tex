\chapter*{Abstract}
\label{chap:abstract}

A host-based intrusion detection system is one possible way to detect attacks on a server. Those systems have been quite effective at finding file based intrusions in the past by creating cryptographic hashes. However, as file sizes have been growing, they began to struggle to execute within a short time frame. Calculating a hash is seen as the only reliable way to find changes to the file system, and with more data, it is taking increasingly long to calculate them. The situation has grown out of proportions because the time to scan now takes so long, that the intrusion detection system can't reliably find intrusions within a useful timespan.

In my thesis, I try to show a different solution to this problem. I created a host-based intrusion detection system that works at a file basis but does not calculate any hash. Instead, it finds intrusions by evaluating the file system attributes like modification time and permissions. This approach risk-based, because it is less reliable, but by increasing the speed the host can be scanned multiple times more often than if hashes get calculated.

Additionally, I want to give forensic investigators a tool where they can have a history of what happened on a file system that takes data from many different snapshots of the host. This way investigators can have a better idea of how attackers tried to establish a foothold and how they can fix that in the future.

This system is based on an open source forensic tool called the sleuth kit, which is a toolkit with much functionality for file system analysis. With my tool, system administrators and forensic investigators have another option to tackle intrusion detection. Taking the risk-based approach can lead to many fast detections that otherwise would not have been detected in time.

