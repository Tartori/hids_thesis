
\section{Security}
\label{sec:Security}

A \gls{hids} is a system that improves the security of a host by detecting \glspl{intrusion}. Thus, it is essential to evaluate the security of such a system. This evaluation is the goal of this section. First, I show some limitations of the \gls{fids}. It has some downsides, and it is necessary to look at them. After that, I describe some risk associated with running the \gls{fids}. Those risks might be associated with all \gls{hids}. Then I give some attack scenarios. Those include some scenarios where the \gls{fids} is better suited than other \gls{hids}, but also some that explicitly attack the limitations and risks. Lastly, I give examples of how to circumvent those attack scenarios. Some might be defeated if we change some small things, while some cannot be avoided.

\subsection{Limitations}
\label{sec:Limitations}

There are some limitations that I list below. I reference those in the attack scenarios and detection sections.

\subsubsection{Changing Attributes}
\label{sec:limitation:chattr}

As already mentioned multiple times, the proposed system doesn't use \glspl{hash} for the detection of \glspl{intrusion}. This approach can be an issue as it means that changed files might not be found. Attributes can be changed maliciously, which could lead the \gls{fids} to miss an \gls{intrusion}.  \cite{chaning:times, changing:attributes}

Even though it is possible to change the \gls{fs} attributes it is not trivial to change all of them. The changing of some leaves traces and for some attributes root access is required. Additionally, the attacker would first needs to remember the exact previous values such that he can correctly reset them.  \cite{chaning:times, changing:attributes}

Lastly, it is possible that the attacker does not change all attributes or that he sets them incorrectly, such that the \gls{fids} can still discover the \gls{intrusion}. Some changes to the attributes leave other traces, such as kernel-level logs. Those can be evaluated by an appropriate tool to find this kind of \gls{intrusion}.

\subsubsection{Non file based \gls{intrusion}}
\label{sec:limitation:nonFileBased}
Another limitation to any \gls{hids} that operates on \gls{fim} is \glspl{intrusion} that are not file-based, which are getting more popular even as \gls{apt}. Many of the biggest threats for web applications, according to \gls{owasp}, are not creating any files on the host \cite{owasp}. There are also other attacks which don't need to use files if they can hijack an already started process. Those attacks are not detectable because the system only works with \gls{fim}.

Those kind of \glspl{intrusion} are getting more frequent and as already said, can't be discovered by only checking the \gls{fs}. However, they leave other traces. They often leave logs in the application because for most of the \gls{owasp} top ten, many trial and error is required. \gls{apt}s are more challenging to find. They leave traces as well in the form of weird network behavior. Because to achieve persistence, they need to redownload the malware after a host has been rebooted. This traffic leads to a perfect opportunity for a \gls{nids}

\subsubsection{Intrusion in non watched location}
\label{sec:limitation:nonWatched}
When the scanner is started it is fed with the paths to scan and which to ignore. If an \gls{intrusion} injects a file in a place that is not watched, it cannot be found. This limitation comes from an incorrect configuration.

For non-watched locations, it is as if the host does not run a \gls{hids}. Maybe the \gls{intrusion} will at some point, write or read something in a watched location, but otherwise, the threat can go unnoticed for a long time. I recommend that at least the scanner part of the \gls{hids} runs on the whole system. Then the attacks can at least be discovered by analyzing the timeline. Additionally, it is recommended to run the investigator on the whole system as well, but maybe with a lower frequency.

\subsubsection{Preexisting \glspl{intrusion}}
\label{sec:limitation:preexisting}
The system can only detect \gls{intrusion} by unexpected changes. If it starts on an already running host, it is possible that this host is already infected. This infection cannot be seen by the system, as long as the files from the \gls{intrusion} stay consistent. 

Preexisting \glspl{intrusion} are hard to detect if they don't alter their files. Luckily, malware behaves like a typical software project. At some point, it needs to update its executable. It might also read or write other files that can trigger an alert. Additionally, if the \gls{fids} is deployed with new machines as well as to the running and infected machines, the new machines are detecting the \gls{intrusion} as soon as the already infected hosts infect the newly deployed ones.

\subsubsection{Shut down \gls{fids}}
\label{sec:limitation:noscan}
The system relies on a running scanner. On a host, this would probably be done by creating a cronjob or some form of sheduled task. If this can be disabled, then the scanner will no longer run, and no further \glspl{intrusion} are detected. Additionally, the investigator could also be stopped. This situation would lead to the same result with the difference, that the database is still written.

Should the scanner be shut down, then the investigator will no longer find any \glspl{intrusion}. This can be detected if the database is checked on new entries from another source. This is easiest done when an external database is used or when the \gls{sqlite} file is copied from the host to an external machine at some points. Also, if logs are configured to be written after each execution, it can be suspicious if those entries are missing or are showing to be evaluating the same runs over and over again. 

This limitation is not highly likely, because the intruder first needs to get administrative access to the host to be able to shut down the cronjob. This is not always easy to achieve, and once the attacker has administrative access, he can do whatever he wants on the system. 

\subsubsection{Delete DB or change Configuration}
\label{sec:limitation:delete}

The intruder could also change the database and remove his traces or change the configuration to not include his files. The configuration file can be seen on the database due to the config hash that is included in each run. The db modification can not be detected, however, it needs to be changed in just the correct moment, after the scanner is completed and before the investigator has started. This timewindow is rather small for an attacker to exploit.

To protect from either attack the configuration and database could be changed in a way that they need administrative privileges to write. This way the attacker first needs to get those privileges and once he got those, he can control the system and do what he wants anyways.

\subsection{Risks}
\label{sec:risk}

Aside from the limitations, there are some risks with running this system. As with the limitations, I give some comment to each risk and talk about how to evaluate them.

\subsubsection{Running unknown code}
\label{sec:risk:unknowncode}

As for any program, it is always a risk associated with the execution of code that is downloaded from the internet. It is possible that it was modified or that it is exploitable. However, this is an \gls{opensource} program. This fact makes it easier to evaluate the code. The same would need to be done for all the dependencies, more specifically for \gls{pytsk} and \gls{tsk}. This is possible but requires much work. However, the \gls{opensource} community is generally well trusted, which might be enough. To protect from alterations to the tool that has been downloaded, hashes can be calculated and the \gls{pgp} signature can be validated.
Additionally, the system behaves very straightforward. It scans the file system and then writes to a database, then reads from the database and creates alerts. This functionality can be monitored. Should the system do something else, it might either have a bug or is being abused.

\subsubsection{File access}
\label{sec:risk:file}

Any \gls{hids} that uses \gls{fim} needs access to the files. Depending on the data that is stored on the host, this might be a cause for concern. The system has full access on any of those files. This cannot be circumvented. What can be done is using only the scanner on this system. Let it write the database file. Then let the investigator run on the data from another host. This way, the system can be monitored to make sure it doesn't do anything else except for writing the database.

\subsubsection{Root access}
\label{sec:risk:root}

The system needs access to the disk image which is mounted. To gain access to that, it needs to run with administrative privileges. With those permissions, the system could theoretically do anything on the host. The best way would be to limit the amount of time that it is executing with those privileges. To do that the scanner and the investigator can be executed in different processes where only the scanner has administrative access. The investigator only needs to read the database and send alerts, for neither it needs administrative privileges.

\subsubsection{Exploiting of the system}
\label{sec:risk:exploiting}

As with any system, it is possible that an attacker can exploit the \gls{fids}. This way he can run arbitrary code on the host. This risk is higher than for most applications since the \gls{fids} runs with administrative privileges.

The best way to limit that risk is, again to limit the parts that run with administrative privileges — this way, the attacker needs to exploit the scanner, which is harder. It is important to note that the \gls{fids} overall does not have a large attack surface. The only data that is read is metadata from the \gls{fs}. There are no open ports, sockets, web connections, or files being read. It also does not need to be running at all times. This already limits the attack surface significantly.

\subsection{Attack Scenarios}
\label{sec:attack_scenarios}

In this section, I describe several attack scenarios. I only explain what the intruder does and how he does it. I might give some description of the host and which processes are running. I also list which limitations or which risks the attacker uses. In section \ref{sec:mittigations} I explain how those attacks can be avoided or detected and what is needed for the detection. Finally, in the discussion, I give a summary of how my system stands against other systems like \gls{aide}. 

\subsubsection{Classic Intrusion with persistance}
\label{sec:attack:classic}

The traditional way to gain a foothold on a host is to write a file on the \gls{fs}. The attacker then executes the file remotely, and the host is compromised. How exactly the attacker was able to upload a file is not relevant for this scenario. The attacker does not do anything special with the file and leave it where it lands. He does not change any attributes or uses any other tactic to hide.

This attack exploits none of the risks and limitations as this is the kind of attack that the system is made to find. 

\subsubsection{Intrusion with persistance, attacker changes all metadata}
\label{sec:attack:changeattr}

The attacker changes a file or creates a file on the host and then resets the attributes in such a way that the system can't find anything wrong.

Risks and limitations:
\begin{itemize}
	\item \nameref{sec:limitation:chattr}
	\item \nameref{sec:limitation:nonWatched}
\end{itemize}

\subsubsection{Intrusion without persistance to use as intermediate host}
\label{sec:attack:nopersistanceintermediatehost}

The attacker exploits a running process. He uses the host to gain access to other machines in the network and does not read or write anything from or to the host. 

Risks and limitations:
\begin{itemize}
	\item \nameref{sec:limitation:nonFileBased}
\end{itemize}

\subsubsection{Intrusion without persistance to exfiltrate data}
\label{sec:attack:nopersistanceexfiltration}

As in scenario named `\nameref{sec:attack:nopersistanceintermediatehost}' the attacker gains access through an exploit of an already running process. However, in this scenario, he does read files of interest, for instance, private keys used for encryption. 

Risks and limitations:
\begin{itemize}
	\item Partially: \nameref{sec:limitation:nonFileBased}
\end{itemize}

\subsubsection{Intrusion without persistance but as \gls{apt}}
\label{sec:attack:nopersistanceapt}

This scenario is similar to the previous two. Here the attacker exploits a process that is already running. He then goes on to use this process to do things that are uncharacteristic for this process. He does neither read nor write files from the host. His goal is to stay hidden for as long as possible. Additionally, he wants to reinfect the host whenever the process is restarted.

Risks and limitations:
\begin{itemize}
	\item \nameref{sec:limitation:nonFileBased}
\end{itemize}

\subsubsection{Attacker exploits \gls{fids} to gain administrative privileges}
\label{sec:attack:exploitforroot}

In this scenario, the attacker can somehow exploit the \gls{fids} to gain administrative privileges. The most likely way is to give it a different configuration or change the code that is already running on the system. 

Risks and limitations:
\begin{itemize}
	\item \nameref{sec:risk:exploiting}
	\item \nameref{sec:risk:root}
	\item \nameref{sec:risk:unknowncode}
\end{itemize}

\subsubsection{Attacker changes the code maliciously}
\label{sec:attack:codechange}

The attacker changes the code that the user downloads. He injects his malicious code in the version. The victim then executes the \gls{fids} and the attacker can take over the host.

Risks and limitations:
\begin{itemize}
	\item \nameref{sec:limitation:preexisting}
	\item \nameref{sec:risk:root}
	\item \nameref{sec:risk:unknowncode}
\end{itemize}

\subsection{Attack Detection}
\label{sec:mittigations}

To detect the attacks described in \nameref{sec:attack_scenarios}, there are multiple things one can do. First, I give some information about how the specific scenarios can be detected if possible. Then there are some general recommendations on what can be done to increase the security of the system or the host.

\subsubsection{Classic Intrusion with persistance}
\label{sec:defense:classic}

This attack scenario is the easiest. The \gls{fids} detects this kind of \gls{intrusion} if the configuration is correct. Compared to other \gls{hids}, this kind of attack is found faster, because the system executes faster by not calculating \glspl{hash}. 

\subsubsection{Intrusion with persistance, attacker changes all metadata}
\label{sec:defense:changeattr}

The \gls{fids} can only find this kind of attack if the attacker takes longer than the system has to scan. It is possible that the \gls{fids} discovers it while the attacker has not yet changed all the attributes. Additionally, it is possible that the attacker did not change all the related attributes. In both cases, the system finds the \gls{intrusion} and alerts it. Should the attacker be fast enough, the system does not find the changed file. Then it depends on how well the attacker hides. If he does not do anything on the \gls{fs}, it is possible that this kind of \gls{intrusion} goes unnoticed for a long time by the \gls{fids}. 

Here a \gls{nids} or a different \gls{hids} that evaluates different parts of the host come into play. A \gls{hids} running on processes or ports could find this \gls{intrusion}. Also, a \gls{nids} can detect the \gls{intrusion} because the attacker needs to communicate to or from the compromised host at some point. \gls{fids} is not suited to find this kind of attack on its own.

\subsubsection{Intrusion without persistance to use as intermediate host}
\label{sec:defense:nopersistanceintermediatehost}

Any \gls{hids} that uses \gls{fim} cannot detect this type of attack. To still detect this kind of attack a \gls{nids} can be used. Because the attacker most likely wants to gain access to other hosts in the network. When there is a \gls{nids} installed, it should be able to detect this traffic and alert them. Another way would be to deploy an additional \gls{hids} that does not use \gls{fim}. This way, it would find the \gls{intrusion} and can alert it.

\subsubsection{Intrusion without persistance to exfiltrate data}
\label{sec:defense:nopersistanceexfiltration}

This type of attack is visible for a \gls{nids}. Large amounts of data are transferred in a way, that might not be usual. This can result in alerts from a \gls{nids}. However, the \gls{fids} can detect this kind of attack as well. As the attacker reads the data, it changes the modified timestamp, which can be watched using the \gls{fids}

\subsubsection{Intrusion without persistance but as apt}
\label{sec:defense:nopersistanceapt}

This type of \gls{intrusion} is not possible to detect by a \gls{hids} that works with \gls{fim}, except the attacker reads or writes files. A \gls{nids} is needed which can detect the \gls{intrusion} because after the infected process is restarted, another host reinfects it. This is the kind of traffic a \gls{nids} should be able to detect.

\subsubsection{Attacker exploits \gls{hids} to gain administrative privileges}
\label{sec:defense:exploitforroot}

For the \gls{fids}, this attack is not detectable, because the attacker changes the system itself. The best way to protect against that is to monitor what it does carefully. The scanner part should only be allowed to read data from the disk image and write it to the database file. The investigator should be executed separately without admin privileges. This way, the attack surface is diminutive. The only apparent means for the exploit is the scanner config. To still find the \gls{intrusion}, the \gls{fids} might already have written the hash of the configuration to the database. By checking the hash, the attacker could be found. 

\subsubsection{Attacker changes the code maliciously}
\label{sec:defense:codechange}

This attack is not detectable from the system, because firstly, the system itself was changed and behaves abnormally. Moreover, because when the system was executed, the \gls{intrusion} already happened. To protect from this form of attack, the system administrator should check the hash of the \gls{hids}. Additionally, the \gls{pgp} signature needs to be checked. This way, an administrator can make sure that the package which is installed has not been altered. Further, the source code can be checked and compiled by the system administrator himself. This would take more work, but the possibility for malicious inclusions are fewer. 

\subsection{General defense}
\label{sec:defense:general}

Generally, it makes sense to use a \gls{nids} in conjunction with any \gls{hids}. Some \glspl{intrusion} are more suitable for detection on the host level and others on a network level. Both make sense, so both should be used. Additionally, it makes sense to use an additional \gls{hids} which is specialized in detecting \glspl{intrusion} on the process, or network activity level. This way, the host is monitored fully and \glspl{intrusion} can be even better detected. For the \gls{fids}  specifically, it makes sense to run the scanner and the investigator separately.
