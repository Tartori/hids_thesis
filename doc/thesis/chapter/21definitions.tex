\chapter{Technical Background}

In this chapter, I provide some technical introduction to relevant topics. Further information is available in the linked resources.

\section{Definitions}

\subsection{Filesystem}
\label{sec:fs}

\Glspl{storagemedia} takes many forms. There are the traditional \gls{hdd} and newer \gls{ssd}. Both are built based on different assumptions and using different technologies. There exist more \glspl{storagemedia} like \gls{cd}, \gls{dvd}, \gls{usb} flash drives and more. \Glspl{storagemedia} operates on blocks of data, which are typicaly 512 bytes or 4 \gls{kb} in size. \cite{bruce:imaging}

This creates a problem when an \gls{os} want to use a \gls{storagemedia} to store data. The \gls{os} generally wants to store files which can be any size from some bytes up to several \gls{gb}. \Glspl{fs} were created that the \gls{os} does not need to be concerned with data blocks. They create a layer of abstraction for the \gls{storagemedia} and offer the \gls{os} a simple and standard way to access files. Another benefit of \glspl{fs} is that they store \gls{metadata} such as timestamps, permissions, and attributes for each file. This data helps the \gls{os} to identify relevant data.

The \gls{fs} has it's own datastructure. It uses so called inodes to store the \gls{metadata} and uses the blocks of the \gls{storagemedia} to store the files themselves. Which \gls{metadata} is available heavily depends on the \gls{fs} used. There are some standard \gls{metadata}, that almost all \gls{fs} support, but there are also more \gls{metadata} that only certain \gls{fs} offer \cite{bruce:imaging}.

\subsection{Cryptographic Hashing Function}
\label{sec:hashing}

A hash function is, in general, a function which takes inputs of unspecified size and generates an output of a fixed size. Hash functions are used widely in computer science. Examples where hashing is needed are programming and databases where they are used to access and add certain data within a data structure easily. By design, hash functions have \glspl{collision}, meaning that multiple inputs generate the same output. This statement must be true if you consider the unlimited input size and limited output size. If there are more inputs than outputs, there must be at least one output value that is assigned to multiple inputs. For data storage and other use cases this is not a huge problem because \glspl{collision} can be handled, and if the hash function has good distribution, \glspl{collision} are unlikely. Additionally, in many systems, a hashing function with weak \gls{collision} resistance is deliberately chosen because it is usually faster to execute \cite{hash:noncrypto, hash:slow}.

In a cryptographic context, this tradeoff cannot be taken. Two big factors play into why not. Firstly, in cryptography, hashes are often used as an assurance that the content of some data has not changed. If \glspl{collision} are easy to find, the data can be altered in ways that result in the same hash, meaning that the hash no longer fulfills the use case. Additionally, monetary incentive can be big. If a \gls{collision} can be provoked, even if challenging, data can be changed. This data could be a legal document or a bank transaction, neither of which we want to change. For those reasons, a cryptographic hash function needs to be highly \gls{collision} resistant. The drawback of \gls{collision} resistant algorithms over simpler hashing functions is the number of operations needed for the calculation, while current hashing functions are performant and secure, they still take some time for much data \cite{crypto}.

One \gls{hash} is called \gls{sha256}. \gls{sha256} is a standard published by the \gls{nist}. It creates a 256-bit output and has not yet been successfully attacked \cite{sha, crypto}. This hashing algorithm is used in the implementation of the \gls{hids} to make sure that the configuration file has not changed between multiple runs.


\subsection{HIDS}
\label{sec:def:hids}

A \gls{hids} works by detecting changes on the local host. It does that by looking at files, processes, configuration, logs, or other indicators. In this thesis, I focus on \gls{fim}. It is important to note that the other origins mentioned are also a valuable source of information. Any of those might have \gls{ioc}. Especially running processes and the configuration can hold relevant data. However, this implementation only covers file-based information due to the lack of resources.


In a \gls{hids} that uses \gls{fim}, all the relevant information comes from unexpected changes to the \gls{fs}. Many hosts do not have any changes on the \gls{fs} except for software updates. Additionally, the \gls{fs} usually contains too much information to cover in a handwritten configuration. Thus, the easiest approach is to compare the current state of the \gls{fs} to a previous one. If it has changed significantly or in an unexpected way, something foul might have happened or might still be happening. The negative side effect of this approach is the false positives that come from legitimately changing the host. Those legitimate changes could be new versions of the webpage that is running, newly updated configuration or updated ssh keypairs. However, those changes are scheduled, and the alerts can then be quickly checked and acknowledged.

A good \gls{hids} should be able to handle such valid changes without change on the configuration or other changes on the system. Additionally, it should be able to find intrusions reliably and in a timely fashion. Maybe, it is too late if the \gls{hids} finds an intrusion a week after the infection. To detect changes on the \gls{fs} reliably prominent \gls{hids} calculate a hash for each file and compares that hash to previous runs. If a \gls{hash} is used, each file always generates a unique hash which can not be forged. The main drawback of using \glspl{hash} is that they take a long time to compute for significant amounts of data. Some implementations thus offer to use \glspl{nonhash}. The obvious drawback of this approach is that a \gls{collision} can be generated and such a file can be altered or replaced without the \gls{hids} noticing.

In the following sections, I present three of the most used \gls{hids}, \nameref{sec:tripwire}, \nameref{sec:aide}, and \nameref{sec:samhain}. They built the main competition of the integration developed in this thesis.

\subsubsection{Tripwire}
\label{sec:tripwire}

In 1992 the first \gls{hids} named \gls{tripwire} was created and publicly released as a free tool. In 1997 the creator of \gls{tripwire} then created the company \Gls{tripwire} Inc. and bought the naming rights for \gls{tripwire}. The free version was monetized, and they released new versions of \gls{tripwire} \cite{Tripwire:Impl, Tripwire:company}. Now they mostly market to enterprise and industrial customer and have multiple products for securing computer systems and networks. \cite{tripwire}

From the documentation that they provide to potential customers, it is unclear how their product works exactly. This approach makes sense because this way, they can keep competitors at bay. However, they have some information about how their product works to attract new customers. They claim to find changed data in real time. Additionally, they seem to be able to know what part of the documents changed. They then use another product of theirs called ChangeIQ to determine if the change was malicious or not. \cite{tripwire:fim:datasheet, tripwire:true:intent}

For this approach to work, they need to create a big database with the content of each file. Then \gls{tripwire} probably is hooked into the kernel to get notified about changes to the \gls{fs} in real time. This approach has benefits over hashing and over checking attributes because it is faster than the former and more reliable than the latter. However, for that to work, the system needs to be running at all times. Additionally, not every \gls{os} supports this kind of kernel level information. I could not find out how they find \glspl{intrusion} should they ever be shut down and restarted at a later time or on \gls{os} on which this information is not available. 

What has to be said, \gls{tripwire} does not directly compete with my solution. \Gls{tripwire} is a commercial product with integration into monitoring and fancy user interfaces. They have much financial incentive to continuously improve their product and have high integration with an extensive suite of security tools aimed at big cooperations. I encourage people working for this type of company to have a look at \gls{tripwire}. However, because it mainly focuses on those big corporate customers, it might be too expensive for smaller teams. My \gls{hids} is more focused on smaller companies, teams within large companies, and participants of the \gls{opensource} community and is entirely free.

Even though the direct comparison might not even be relevant, my system has some advantages that I want to list here. It does not need a large amount of storage. For tripwire to work, it needs to store files entirely or partially to compare them to previous states accurately. This file storage might not be a problem for small files, but it does take much space if many machines are scanned or the amount of small files is extensive. My system only saves \gls{metadata}, which is considerably less space.

Additionally, my system does not need to run at all times. This approach reduces the load, which is especially essential on high-performance machines, particularly if they have a lot of read-write operations. One final advantage is the fact that my system is \gls{opensource}. With that, any potentially interested party can look at the source code and see if the product is appealing to them. Additionally, they can check the source code for security issues and can find out exactly how the system works. This information is not available for \gls{tripwire} customer. They have to trust the company behind it.


\subsubsection{AIDE}
\label{sec:aide}

\Gls{aide} is an \gls{opensource} alternative to \gls{tripwire}. It was created after \gls{tripwire} went commercial. As \gls{aide} is \gls{opensource}, it is easier to find information about how it works, which I detailed below. \cite{aide:totherescue, aide:github}

When \gls{aide} is first executed, it generates a reference database. This database contains all the information for each file that is within a path of interest. Depending on the config it contains more or less information, including \glspl{hash}. Each subsequent execution then compares the files found to this initially created database. The database needs to be updated manually if the system has a scheduled change. It generates a log of all the changes and alerts all of the changes at once. Both the configuration and the database are usually stored on the file system that is scanned. \cite{aide, aide:doc}

\Gls{aide} can compare multiple \glspl{hash} and \glspl{nonhash} and some \gls{fs} \gls{metadata}. It runs on many modern Unix system \cite{aide} and uses \gls{pgp} keys to sign their releases. It has a strong community behind it, but only offers a \gls{cli} and has no fancy user interface and is written in the programming language C. \cite{aide:github} 

\Gls{aide} also features a rich configuration. It is based on rules. There are some preconfigured rules for each hashing function and for the checked \gls{fs} \gls{metadata}. Those rules can then be combined in new rules to create custom settings. Finally, the configuration offers the possibility to specify paths which should be scanned with the rules that should be applied. This way, different directories can be scanned with different settings. \cite{aide:conf}

\subsubsection{Samhain}
\label{sec:samhain}

\Gls{samhain} is another \gls{opensource} \gls{hids}. In comparison to \gls{aide} it has a company backing it which offers support for customers. However, other than \gls{tripwire} it is not necessary to use their services to run \gls{samhain}. For that, they have a lot of publicly available documentation and community support. \cite{samhain:company}

\Gls{samhain} has a big focus on centralized management of the \gls{hids}. All hosts run the system and report their findings to a single instance which is used for management and monitoring. The hosts also get the configuration from the server. For scanning, it has two modes. Firstly, it can scan the system similarly to \gls{aide} by going through all the files and mostly generating hashes. However, it can also create a hook in the Linux kernel to find changes live. It uses both approaches to find \glspl{intrusion}. Contrary to \gls{aide} \gls{samhain} offers more functionality than \gls{fim} and can check open ports, processes and more. 

Compared to my system, it can leverage the kernel integration to find \glspl{intrusion} faster. However, it still relies on hashing for scanning parts of the \gls{fs}, especially on non-Linux systems. This part of \gls{samhain} is improved by my system. Overall, \gls{samhain} offers more functionality than \gls{fim}, which my system currently does not. For kernel integration, it needs to be running at all times. As already discussed in the tripwire section, this can be a drawback.

\subsection{NIDS}
\label{sec:def:nids}

Compared to a \gls{hids}, a \gls{nids} can be used to detect \glspl{intrusion} from multiple hosts at the same time. \gls{nids} are used to analyze anomalies on the network. An \gls{anomaly} on the network might be an unexpected session from a server which usually doesn't communicate with the internet or unusually large or frequent traffic. Certain \glspl{intrusion} can be tracked by analyzing the content of the packages. Some forms of \glspl{intrusion} are easier to detect on the network. Especially if they extract much data or are very aggressive at spreading within a network. The main advantage here is that multiple sources can be combined on the network level. Another advantage of \gls{nids} is that current \gls{malware} almost always contains some network communication \cite{Malware:Behaviour,nids}.

However, a \gls{nids} doesn't only have advantages. Certain kinds of attacks can only be detected on the network with much difficulty. For instance, a corrupted file upload which allows uploading of any file type can hardly be found by checking the network traffic. For best results, both types of \gls{ids} can be used in conjunction to detect attacks. 

In this thesis, \gls{nids} are not in the center. The implementation doesn't look at network traffic or mimic other \gls{nids} functionality. However, as mentioned above, I highly encourage everyone to use an appropriate \gls{nids} to improve the chance to find an \gls{intrusion}. 

\subsection{The Sleuth Kit}
\label{sec:tsk}

\gls{tsk} is an \gls{opensource} toolkit used to investigate disk images. It is based on \gls{tct} \cite{tct} and contains multiple command line interfaces and an \gls{api} for various purposes \cite{tsk, tsk:about}. It is mainly written in the programming language C, runs on most \gls{os} and is used to analyze numerous \glspl{fs}. It is heavily used in forensic investigations to find deleted or corrupted files and for other information gatherings on a disk image.

It contains many tools which are used for specific purposes. Those include analysis of partitions, blocks, inodes, filenames, journaling \gls{fs}, timelines, and more. In this thesis, I am mainly concerned with the tool that operates on filenames. This tool is called fls, which I described in the next section. It also contains a mactime utility which is described in section \nameref{sec:mactime}.

\subsubsection{fls}
\label{sec:fls}

Fls can be used to access directories, files, and their attributes. With it, the directories can be displayed recursively, and for each file, the attributes can be printed to the console. \cite{tsk:fls} 

This tool also can be accessed via the \gls{api} and is used in the \gls{hids} implementation to retrieve the directories, files, and attributes. The process of how this is achieved is explained in section \ref{sec:Scanner}.

\subsubsection{mactime}
\label{sec:mactime}

Mactime is a utility for timeline creation. It takes data in a format they call body file and convert it to a ascii timeline. \cite{tsk:mactime, tsk:bodyfile}

\subsection{Python}
\label{sec:python}

Python was the programming language of choice for this product. Python offers many advantages over other languages. Some of those are:

\begin{itemize}
    \item Platform independence
    \item Good community support in the forensic community
    \item Active library for \gls{tsk} named \gls{pytsk}
    \item Small overhead
    \item Easy for prototyping
    \item Easy to read
\end{itemize}

This decision was not made very lightly. Other programming languages were considered. For more information on those refer to appendix section \ref{sec:decisions:language}.

\subsection{pytsk3}
\label{sec:pytsk3}

Pytsk3 is the library mentioned above that creates bindings for python to the \gls{api} of \gls{tsk}. As \gls{tsk}, this library is \gls{opensource} and is still active. It is hosted on \gls{github} and offers most of the functionality of the \gls{tsk} \gls{api}. Pytsk3 is used extensively in the scanner part of the implementation. For further information refer to section \ref{sec:Scanner}.
