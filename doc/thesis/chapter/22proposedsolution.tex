
\section{Proposed solution - FIDS}
\label{sec:fids}

The main principle behind a \gls{hids} is the detection of changed files. As already discussed in section \ref{sec:def:hids}, most tools rely on the calculation of hashes. This process is generally a good approach since changes can be found very reliably; however, as already mentioned, it can drastically hinder the performance of the \gls{hids}. Sadly the actual performance lost cannot be clearly stated, as it heavily depends on what hardware is in use. However, considering the computational overhead of calculating a \gls{hash} it is clear that the time it takes grows with bigger \glspl{fs}. As \gls{storagemedia} has grown from a \gls{mb} to \gls{tb} so has the requirement to store more data. Calculating hashes over such big systems is not viable as it can take a very long time. \cite{hash:slow, hash:veryslow, hash:speed}

The name of this proposed solution is the \gls{fids}

\subsection{Time issues}

In this thesis, I propose a different approach, the \gls{fids}. Ignoring the hashing and the file content and focus on the \gls{fs} \gls{metadata} instead. This approach should be sufficient to find intrusions and thanks to not calculating hashes, the \gls{fids} can execute more often. If a conventional \gls{hids} might take several days to complete a run on a \gls{fs} with multiple \gls{tb} of data, the \gls{fids} would take some minutes to hours. It could then be run multiple times a day and find new and changed files within a short amount of time. It is possible that the \gls{fids} misses some changes if the attacker can change all the relevant \gls{metadata} before the system rechecks the file. However, this risk is small in comparison to not finding an intrusion for multiple days. It is also possible to scan highly critical sections of a system with a traditional \gls{hids} and the rest with the \gls{fids}. This way, one has both advantages. The whole system can be checked in a timely fashion with the \gls{fids}, and for certain smaller parts of the system, a general \gls{hids} can detect changes by using \glspl{hash} like \gls{sha256}.

The \gls{fids} uses \gls{tsk} via \gls{pytsk} to extract the \gls{fs} \gls{metadata}. The main advantage that this gives is the interoperability with various \glspl{fs}. Additionally, by directly accessing the attributes from the \gls{fs}, no \gls{metadata} is changed. The files themselves are never touched. Furthermore, it can also be used to get the files and attributes of an image that has been extracted or of a virtual machine. 

\subsection{Investigation capabilities}
\label{sec:investigation:capabilities}

Another improvement upon existing \gls{hids}, is that investigation capabilities are built in from the start. Traditional \gls{hids} are good at finding intrusions and alerting them. However, they don't offer much support for the investigation of the incident. They don't have information beyond what was configured. This lack of information makes sense if most of the information that is used comes from calculated \glspl{hash}, because storing the hash value does not help. In such a system, if something was incorrectly configured, an investigator does not have any useful output to gain further knowledge. They generate alerts and not much beyond that.

The \gls{fids} stores all the available \gls{metadata} for each scanned file for every run. This way, an investigator can use this database to gain more information about how the attacker proceeded. He can look at the changes of permissions, modification of files, and more, even if the system did not find any intrusions. It also supports the generation of a timeline out of this data to form an expansive view on what happened when on the \gls{fs}. 

\subsection{Flexibility}

\Gls{aide} works by comparing each execution to the initial one. This approach is practical as it detects one intrusion multiple times. However, it generates many messages, and users are then less likely to take them seriously because of false positives. Additionally to that, after a legitimate change to the system, \gls{aide} always generates alerts until the initial run is reset. This behavior can lead to undetected intrusions a short time after an upgrade, which is often used by attackers to gain a foothold because of the high number of alerts created by the upgrade. Thus, it is possible that an intruder can gain access shortly after an update which is alerted by \gls{aide} but ignored because of the false positives. \Gls{aide} is only used as an example here, other \gls{hids} might work similarly and would give the same result.

The \gls{fids} does not have this issue, at least not as strongly. As all the data gets collected for each run, it makes sense to compare each run to the previous. This approach has the benefit of legitimate changes being considered to be acceptable one run after they happened. This acceptance results in overall fewer messages, which increases the importance of each. If it is required that the system should always compare against one specific run, this could also be done. The \gls{fids} only needs to be configured to not check against the latest, but a specific, predefined run. Only the configuration then needs to be changed to update this predefined run. 

\section{Scope}

In this project I create a \gls{hids} which uses \gls{tsk} to detect changes named \gls{fids}. It covers the three main changes discussed in section \ref{sec:fids}. The \gls{fids} uses a \gls{sql} database to store the executed runs. It includes user documentation and this thesis documentation. It also contains the creation of a timeline for investigation purposes mentioned in section \ref{sec:investigation:capabilities}.

Out of scope are extensive alerting functionality and extensive example configurations for commonly used \gls{os} and tools. Furthermore, any big data analysis of the runs, while very interesting, are also out of scope.

