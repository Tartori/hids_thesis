\section{Examples}
\label{sec:Exaples}

In this section, I give some general use cases and the relevant configuration. I mainly use the Apache web server as an example. There are some assumptions on the directories and contents for both scenarios. Additionally, I assume that the file system is accessible through `/dev/sda1'.

For additional help, there is also a user-documentation in appendix section \ref{sec:userdocu}

\subsection{Webpage hosted in Apache}

For those examples, I work under the assumption that the Apache web server is installed and has an `htdocs' directory that is located in `/var/www/'. I assume that there is a webpage running with some PHP files. It also has HTML, CSS, javascript, and image files in a static folder. Then it has a log directory where the webpage itself stores all the logs and an upload directory where the user can upload all kind of files. 4 types of files are watched, some more and some less closely. I give Example Configurations each type. For the scanner, the configuration in listing \ref{lst:example:scanner} is appropriate.

\begin{lstlisting}[language=yaml, numbers=left, caption=Example Scanner Configuration, label=lst:example:scanner]

sqlite:
    filename: apache.db

scan:
    image_path: /dev/sda1
    scan_paths: 
        [
            "/var/www/htdocs/",
        ]

\end{lstlisting}

\subsubsection{PHP files}

The web server has some PHP files in different directories. Those are used for various purposes, but they should never change, except for an update to the webpage. All of them lie in some subdirectory of the `htdocs' directory. Thus they all are covered in the scan from the configuration in listing \ref{lst:example:scanner}. 

The tricky part is that we have multiple different PHP files in different directories. The configuration should cover that even if the exact location and folder structure changes a bit and new folders are added. As the scanner automatically scans recursively, the exact folder structure does not matter. The files are found when they are in the `htdocs' directory. As the files should not change but might be accessed at any time, watching everything should be what we want. Especially noteworthy are the modified, changed, and created times. Also important are size, path, name, and inode.

This much for the rule, the investigation is then relatively easy. There is not a particular path that has to be watched since the scanner was configured only to scan `htdocs'. By using the same database, the investigator only validates this data as well. Important though is the whitelist. There is an upload directory in which we expect changes. Thus a file name regex of `.php\$' successfully narrows the files to only the required ones.

An example investigator configuration is listed in listing \ref{lst:example:php}. For this configuration, I did not use the preconfigured rules.


\begin{lstlisting}[language=yaml, numbers=left, caption=Example PHP File Configuration, label=lst:example:php]
sqlite:
    filename: apache.db

investigator:
    same_config: True
    rules: 
        - name: equal_but_accessible
        equal:
            - meta_creation_time
            - meta_creation_time_nano
            - meta_modification_time
            - meta_modification_time_nano
            - meta_changed_time
            - meta_changed_time_nano
            - meta_size
            - path
            - meta_addr
            - name_name
        greater:
            - meta_access_time
        investigations:
            - fileregexwhitelist: '.php$'
              rules:
                - equal_but_accessible
\end{lstlisting}

\subsubsection{Upload directory}

As already mentioned, there is an upload directory. Creating a correct configuration for this directory is more challenging than for the PHP files since any files might get uploaded and deleted from there all the time. The critical part here is that no one can upload a PHP file. If this was possible, an intruder could create any mischief. This \gls{intrusion} already gets caught by the rule from listing \ref{lst:example:php} because it already scans for PHP files in the whole directory. It is still viable to create an additional rule that only catches those types of \glspl{intrusion}. Maybe this investigation is then executed more often than the other. Additionally, this configuration also checks for additional properties and alerts a PHP file in the upload directory until it has been deleted

For the rule we need to add the special rules `new\_files\_ok' and `deleted\_files\_ok' since both are valid in this directory. Then generally greater access and modification times are ok. For the investigation, a file regex blacklist is added. Additionally only the upload directory needs to be searched. With that, valid PHP files in other directories are easily excluded. The example configuration is shown in  listing \ref{lst:example:upload} 

\begin{lstlisting}[language=yaml, numbers=left, caption=Example Upload Directory Configuration, label=lst:example:upload]
    sqlite:
        filename: apache.db
    
    investigator:
        same_config: True
        rules: 
            - name: upload
              greater:
                - meta_modification_time
                - meta_changed_time
                - meta_access_time
            investigations:
                - fileregexblacklist: '.php$'
                  paths: [
                    "/var/www/htdocs/upload/"
                  ]
                  rules:
                    - upload
                    - new_files_ok
                    - deleted_files_ok

\end{lstlisting}

\subsubsection{HTML, CSS, Javascript and images}

The webpage also has some static context, namely HTML, CSS, Javascript, and image files. Those are in a directory called `static'. Those files are like the PHP files; they should not change unless someone creates an update to the page. In this static folder, only those files are acceptable, in the configuration, this results in a blacklist for those file types which is negated. This way, files that don't equal these file types are alerted. Additionally, the previously used rule named `equal\_but\_accessible' should do the trick to find modified files. An example configuration is shown in listing \ref{lst:example:static} 

\begin{lstlisting}[language=yaml, numbers=left, caption=Example Static Files Configuration, label=lst:example:static]
    sqlite:
        filename: apache.db
    
    investigator:
        same_config: True
        rules: 
            - name: equal_but_accessible
              equal:
                - meta_creation_time
                - meta_creation_time_nano
                - meta_modification_time
                - meta_modification_time_nano
                - meta_changed_time
                - meta_changed_time_nano
                - meta_size
                - path
                - meta_addr
                - name_name
              greater:
                - meta_access_time
        investigations:
            - fileregexblacklist: '.(js|css|png|ts|jp(e)?g|htm(l)?)$'
              blacklist_inverted: true
              paths: [
                "/var/www/htdocs/static/"
              ]
              rules:
                - equal_but_accessible

\end{lstlisting}

\subsubsection{Log directory}

With that, the upload directory remains. Log files usually grow in size until they reach a certain limit. Then they stay static, and a new file is created, which then again grows. To create a configuration for that, it is important to check the size for growing. Additionally, all the timestamps should also only grow. One important detail is the `file\_rename\_ok' rule as the name of rolled over log files might change. Also, the `new\_files\_ok' rule needs to be added. The example configuration is available in listing \ref{lst:example:log}

\begin{lstlisting}[language=yaml, numbers=left, caption=Example Log Directory Configuration, label=lst:example:log]
    sqlite:
        filename: apache.db
    
    investigator:
        same_config: True
        rules: 
            - name: logs
              equal:
                - meta_creation_time
                - path
                - meta_addr
              greater:
                - meta_modification_time
                - meta_changed_time
                - meta_size
                - meta_access_time
        investigations:
            - paths: [
                "/var/www/htdocs/logs/"
              ]
              rules:
                - logs
                - file_rename_ok
                - new_files_ok

\end{lstlisting}

\subsubsection{Complete configuration}

The partial configurations can then be combined into one bigger configuration. Some rules can be reused, which makes everything a bit easier. The complete configuration is shown in listing \ref{lst:example:complete}.

\begin{lstlisting}[language=yaml, numbers=left, caption=Complete Example Configuration, label=lst:example:complete]
    sqlite:
        filename: apache.db
    
    scan:
        image_path: /dev/sda1
        scan_paths: 
        [
            "/var/www/htdocs/",
        ]

    investigator:
        same_config: True
        rules: 
            - name: upload
              greater:
                - meta_modification_time
                - meta_changed_time
                - meta_access_time
            - name: equal_but_accessible
              equal:
                - meta_creation_time
                - meta_creation_time_nano
                - meta_modification_time
                - meta_modification_time_nano
                - meta_changed_time
                - meta_changed_time_nano
                - meta_size
                - path
                - meta_addr
                - name_name
              greater:
                - meta_access_time
            - name: logs
              equal:
                - meta_creation_time
                - path
                - meta_addr
              greater:
                - meta_modification_time
                - meta_changed_time
                - meta_size
                - meta_access_time
        investigations:
            - fileregexwhitelist: '.php$'
              rules:
                - equal_but_accessible
            - fileregexblacklist: '.php$'
              paths: [
                "/var/www/htdocs/upload/"
              ]
              rules:
                - upload
                - new_files_ok
                - deleted_files_ok
            - fileregexblacklist: '.(js|css|png|ts|jp(e)?g|htm(l)?)$'
              blacklist_inverted: true
              paths: [
                "/var/www/htdocs/static/"
              ]
              rules:
                - equal_but_accessible
            - paths: [
                "/var/www/htdocs/logs/"
              ]
              rules:
                - logs
                - file_rename_ok
                - new_files_ok

\end{lstlisting}

\subsection{SSH private keypairs}

Another example for the \gls{fids} is the tracking of private and public key pairs. Usually, the web server reads them when it is started. Afterward, it caches them in the memory, and the actual files should not be reread. This situation causes an updated access time to be an \gls{ioc} of a possible \gls{intrusion}. For this scenario, I assume that the key pair is located in the directory `/var/www/.ssh/'. An example configuration is shown in listing \ref{lst:example:keys}.

\begin{lstlisting}[language=yaml, numbers=left, caption=Example SSH Keypair Configuration, label=lst:example:keys]
    sqlite:
        filename: ssh.db
        
    scan:
        image_path: /dev/sda1
        scan_paths: 
        [
            "/var/www/.ssh/",
        ]

    investigator:
        same_config: True
        rules: 
            - name: equal_not_accessible
              equal:
                - meta_creation_time
                - meta_creation_time_nano
                - meta_modification_time
                - meta_modification_time_nano
                - meta_changed_time
                - meta_changed_time_nano
                - meta_access_time
                - meta_access_time_nano
                - meta_size
                - path
                - meta_addr
                - name_name
        investigations:
            - rules:
                - equal_not_accessible

\end{lstlisting}

\subsection{Timeline Creation}

The timeline creator works by taking the database output of the scanner and transforming the file data into a time machine format used by tools like mactime. For creating a timeline the \gls{fids} offers a command line flag `-m'. For the configuration only the sqlite filename is needed, as shown in listing \ref{lst:example:timelineconf}. In listing \ref{lst:example:exampletimeliner} I show how the timeliner can be used and how it's output can create a timeline with the mactime utility.

\begin{lstlisting}[language=yaml, numbers=left, caption=Database Config, label=lst:example:timelineconf]
  sqlite:
      filename: apache.db

\end{lstlisting}

\begin{lstlisting}[language=bash, numbers=left, caption=Timeliner Example, label=lst:example:exampletimeliner]

\λ python fids -m
\λ python fids -m | mactime -d
\λ python3 -m fids -m | mactime -d | grep config.yaml
\end{lstlisting}